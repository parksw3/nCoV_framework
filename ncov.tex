\documentclass[12pt]{article}
\usepackage[top=1in,left=1in, right = 1in, footskip=1in]{geometry}
\usepackage{url}
\usepackage{graphicx}
%\usepackage{adjustbox}

\usepackage{xcolor}
\usepackage{lineno}\renewcommand\thelinenumber{\color{gray}\arabic{linenumber}}
% \linenumbers

\usepackage{pdflscape}

\usepackage{afterpage}

\newcommand{\eref}[1]{Eq.~\ref{eq:#1}}
\newcommand{\fref}[1]{Fig.~\ref{fig:#1}}
\newcommand{\Fref}[1]{Fig.~\ref{fig:#1}}
\newcommand{\sref}[1]{Sec.~\ref{#1}}
\newcommand{\frange}[2]{Fig.~\ref{fig:#1}--\ref{fig:#2}}
\newcommand{\tref}[1]{Table~\ref{tab:#1}}
\newcommand{\tlab}[1]{\label{tab:#1}}
\newcommand{\seminar}{SE\mbox{$^m$}I\mbox{$^n$}R}

\usepackage{amsthm}
\usepackage{amsmath}
\usepackage{amssymb}
\usepackage{amsfonts}

%\usepackage{lineno}
%\linenumbers

\usepackage[pdfencoding=auto, psdextra]{hyperref}

\usepackage{natbib}
\bibliographystyle{chicago}
\date{\today}

\usepackage{xspace}
\newcommand*{\ie}{i.e.\@\xspace}

\usepackage{color}

\usepackage{xspace}
\newcommand{\Rx}[1]{\ensuremath{{\mathcal R}_{#1}}\xspace}
% https://tex.stackexchange.com/questions/86565/drawbacks-of-xspace
% avoid double-\xspace
\newcommand{\Ro}{\ensuremath{{\mathcal R}_{0}}\xspace}
\newcommand{\RR}{\ensuremath{{\mathcal R}}}
\newcommand{\Rhat}{\ensuremath{{\hat\RR}}}
\newcommand{\tsub}[2]{#1_{{\textrm{\tiny #2}}}}

\newcommand{\comment}[3]{\textcolor{#1}{\textbf{[#2: }\textsl{#3}\textbf{]}}}

\newcommand{\rev}[1]{\comment{red}{REV}{#1}}

\newcommand{\swp}[1]{\comment{magenta}{SWP}{#1}}
\newcommand{\jd}[1]{\comment{magenta}{JD}{#1}}
\newcommand{\djde}[1]{\comment{magenta}{DJDE}{#1}}
\newcommand{\bmb}[1]{\comment{magenta}{BMB}{#1}}
\newcommand{\dc}[1]{\comment{magenta}{DC}{#1}}
\newcommand{\jsw}[1]{\comment{magenta}{JSW}{#1}}
\newcommand{\mli}[1]{\comment{magenta}{MLi}{#1}}
\newcommand{\new}[1]{\textcolor{blue}{#1}}

\begin{document}

\begin{flushleft}{
	\Large
	\textbf\newline{
		Reconciling early-outbreak estimates of the basic reproductive number and its uncertainty: framework and applications to the novel coronavirus (SARS-CoV-2) outbreak
	}
}
\newline
\\
Sang Woo Park\textsuperscript{1,*}
Benjamin M.\ Bolker\textsuperscript{2,3,4}
David Champredon\textsuperscript{5}
David J.\,D.\ Earn\textsuperscript{3,4}
Michael Li\textsuperscript{2}
Joshua S.\ Weitz\textsuperscript{6, 7}
Bryan T.\ Grenfell\textsuperscript{1,8,9}
Jonathan Dushoff\textsuperscript{2,3,4,*}
\\
\bigskip
\textbf{1} Department of Ecology and Evolutionary Biology, Princeton University, Princeton, NJ, USA
\\
\textbf{2} Department of Biology, McMaster University, Hamilton, ON, Canada
\\
\textbf{3} Department of Mathematics and Statistics, McMaster University, Hamilton, ON, Canada
\\
\textbf{4} M.\,G.\,DeGroote Institute for Infectious Disease Research, McMaster University, Hamilton, ON, Canada
\\
\textbf{5} Department of Pathology and Laboratory Medicine, University of Western Ontario, London, Ontario, Canada
\\
\textbf{6} School of Biological Sciences, Georgia Institute of Technology, Atlanta, GA, USA
\\
\textbf{7} School of Physics, Georgia Institute of Technology, Atlanta, GA, USA
\\
\textbf{8} Division of International Epidemiology and Population Studies, Fogarty International Center, National Institutes of Health, Bethesda, MD, USA
\\
\textbf{9} Woodrow Wilson School of Public and International Affairs, Princeton University, Princeton, NJ, USA
\\
\bigskip

*Corresponding authors: swp2@princeton.edu and dushoff@mcmaster.ca
\end{flushleft}

\pagebreak

\section*{Abstract}
A novel coronavirus (SARS-CoV-2) has recently emerged as a global threat. 
As the epidemic progresses, disease modelers continue to focus on estimating the basic reproductive number \Ro\ --- the average number of secondary cases caused by a primary case in an otherwise susceptible population.
The modeling approaches and resulting estimates of \Ro vary widely, despite relying on similar data sources.
Here, we present a novel statistical framework for comparing and combining different estimates of \Ro across a wide range of models by decomposing the basic reproductive number into three key quantities: the exponential growth rate $r$, the mean generation interval $\bar G$, and the generation-interval dispersion $\kappa$.
We then apply our framework to early estimates of \Ro for the SARS-CoV-2 outbreak.
We show that many early \Ro estimates are overly confident.
Our results emphasize the importance of propagating uncertainties in all components of \Ro, including the shape of the generation-interval distribution, in efforts to estimate \Ro at the outset of an epidemic.

\section*{Keywords}

SARS-CoV-2, COVID-19, novel coronavirus, basic reproductive number, generation interval, Bayesian multilevel model

\section{Introduction}

Since December 2019, a novel coronavirus (SARS-CoV-2) has been spreading in many parts of the world \citep{pneumonia}.
Although the virus is likely to have originated from animal hosts \citep{andersen2020proximal}, the ability of SARS-CoV-2 to directly transmit between humans, particularly without symptoms, has posed a greater threat for its spread \citep{he2020temporal}.
As of April 23, 2020, more than 2 million cases of the coronavirus disease 2019 (COVID-19) have been confirmed internationally \citep{who94}.

As SARS-CoV-2 began to spread in other parts of China, outside Hubei province, as well as in other countries, many analyses of the outbreak were published as pre-prints (e.g., \cite{bedfordncov, imaincov, liuncov, majumderncov, readncov, zhaoncov}) and in peer-reviewed journals (e.g., \cite{li2020early, riou2020pattern, wu2020nowcasting, zhao2020preliminary}).
These analyses particularly focused on estimating the basic reproductive number \Ro\ --- the average number of secondary cases generated by a primary case in a fully susceptible population \citep{anderson1991infectious, diekmann1990definition} --- in order to assess the pandemic potential of SARS-CoV-2.
Such rapid dissemination of the early analyses played an important role in shaping the response to the outbreak \citep{majumder2020early}.

We commend these researchers for their timely contribution and those who made the data publicly available.
However, the estimates of \Ro from different research groups (as well as the associated degrees of uncertainty) vary significantly even though most analyses relied on reports of confirmed cases from China, particularly from the Hubei province.
Understanding the differences between \Ro estimates is necessary as \Ro provides information about the level of intervention required to interrupt transmission \citep{anderson1991infectious}, and about the potential final size of the outbreak \citep{anderson1991infectious, ma2006generality}.
It can be difficult to compare a disparate set of estimates of \Ro when the estimation methods and their underlying assumptions vary widely.

Here, we show that a wide range of approaches to estimating \Ro during the exponential growth period can be understood and compared in terms of estimates of three quantities: the exponential growth rate $r$, the mean generation interval $\bar G$, and the generation-interval dispersion $\kappa$.
The generation interval, defined as the interval between the time when an individual becomes infected and the time when that individual infects another individual \citep{svensson2007note}, characterizes the relationship between $r$ and \Ro \citep{wearing2005appropriate, roberts2007model, wallinga2007generation, park2019practical};
therefore, estimates of \Ro from different models directly depend on their assumptions about the generation-interval distribution and the exponential growth rate.
Early in an epidemic, information is scarce and there is inevitably a great deal of uncertainty surrounding both case reports (affecting the estimates of the exponential growth rate) and contact tracing (affecting the estimates of the generation-interval distribution).
Ignoring these uncertainties inevitably leads to strong conclusions.
We suggest that disease modelers should make sure their assumptions about these three quantities are clear and reasonable, and that estimates of uncertainty in \Ro should propagate error from all three sources \citep{elderd2006uncertainty}.

We present a statistical framework for averaging across estimated basic reproductive number \Ro from multiple studies and apply the method to seven disparate models published online as pre-prints between January 23--26, 2020 that estimated \Ro for the SARS-CoV-2 outbreak \citep{bedfordncov, imaincov, liuncov, majumderncov, readncov, riouncov, zhaoncov}.
Previous studies have directly calculated the average of reported \Ro values but such methods undermine differences in underlying model assumptions and statistical methods \citep{majumder2020early, liu2020reproductive}.
Instead, we decompose estimated \Ro into three key quantities ($r$, $\bar G$, and $\kappa$) and calculate the average (pooled estimates) of these key parameters.
Calculating \Ro based on the pooled estimates allow us to appropriately average across the uncertainties present in modeling approaches and their underlying assumptions.
We use these pooled estimates to illustrate the importance of propagating different sources of error, particularly uncertainty in both the growth rate and the generation interval.
We also use our framework to tease apart which assumptions of these different models led to their different estimates and the associated degrees of uncertainty.
Despite the availability of more recent and/or updated estimates of \Ro, we restrict ourselves to the estimates above in order to focus on the resolution of uncertainty in the earliest stages of an epidemic.

\section{Methods}

\subsection{Description of the studies}

\afterpage{%
\clearpage
    \begin{landscape}
\begin{table}[!th]
\begin{center}
\scriptsize
\begin{tabular}{l|p{3.5cm}|p{3.5cm}|p{3.5cm}|p{2.4cm}|p{2.7cm}|l}
 & Model & Data & Basic reproductive\newline number \Ro & Mean generation\newline interval $\bar G$ (days) & Generation-interval\newline dispersion $\kappa$ & \\
\hline
Study 1 & Deterministic branching process model & Estimated total number of cases in Wuhan City, China\newline (Jan 18, 2020) & 1.5--3.5 & 10 & 1 & \cite{bedfordncov} \\
\hline
Study 2 & Stochastic branching process model & Estimated total number of cases in Wuhan City, China\newline (Jan 18, 2020)  & 2.5 (1.5--3.5)$^\ast$ & 8.4 & unspecified$^\dagger$ & \cite{imaincov} \\
\hline
Study 3 & Poisson offspring distribution model & Confirmed cases from China and other countries\newline (Dec 29, 2019--Jan 23, 2020) & 2.92 (95\% CI: 2.28--3.67) & 8.4 & 0.2 & \cite{liuncov} \\
\hline
Study 4 & Metapopulation Susceptible-Exposed-Infected-Recovered (SEIR) model & Reported cases from China\newline (Jan 1--21, 2020) & 3.8 (95\% CI: 3.6--4.0) & 7.6 & 0.5 & \cite{readncov} \\
\hline
Study 5 & Stochastic branching process model & Estimated total number of cases in Wuhan City, China\newline (Jan 18, 2020) & 2.2 (90\% CI: 1.4--3.8) & 7--14 & 0.5 & \cite{riouncov} \\
\hline
Study 6 & Exponential growth model & Reported cases from China\newline (Jan 1--21, 2020) & 5.47 (95\% CI: 4.16--7.10)$^\ddagger$ & 7.6--8.4 & 0.2 & \cite{zhaoncov} \\
\hline
Study 7 & Incidence Decay and Exponential Adjustment (IDEA) model & Reported cases from Wuhan City, China\newline (Dec 8, 2019--Jan 26, 2020) & 2.0--3.1 & 6--10 & 0 & \cite{majumderncov} \\
\hline
\end{tabular}
\end{center}
\caption{
\textbf{Summary of the models used and reported estimates of the basic reproductive number and the assumptions about the generation-interval distributions.}
}
Model details, estimates of \Ro, and their assumptions about the shape of the generation interval distributions were collected from 7 studies.
Generation-interval dispersion represent the squared coefficient of variations in generation intervals.
$^\ast$These intervals reflect \Ro values for best and worst scenarios. We treat these intervals as a 90\% credible interval in our analysis.
$^\dagger$We assume $\kappa = 0.5$ in our analysis.
$^\ddagger$The authors presented \Ro estimates under different assumptions regarding the reporting rate; we use their baseline scenario in our analysis to remain consistent with other studies, which do not account for changes in the reporting rate.
\end{table}
\end{landscape}
\clearpage
}

We gathered information on estimates of \Ro and their model assumptions from 7 articles that were published online between January 23--26, 2020.
Five studies \citep{liuncov, majumderncov, readncov, riouncov, zhaoncov} were uploaded to pre-print servers (bioRxiv, medRxiv, and SSRN); one report was posted on the web site of Imperial College London \citep{imaincov}; and one report was posted on \url{nextstrain.org} \citep{bedfordncov}.
Model details are summarized in Table 1.

\subsection{Comparing model assumptions}

Despite a wide range of models considered across Study 1--7, all of them assume that the epidemic grows exponentially in the beginning.
While the IDEA model (used in Study 7) includes a discount parameter $d$ that allows the model to deviate from the exponential growth when $d \neq 0$ \citep{fisman2013idea}, Study 7 estimates $d=0$ across all parameters they consider.
When the epidemic is growing exponentially, the estimated basic reproductive number entirely depends on the exponential growth rate $r$ and the \emph{intrinsic} generation-interval distribution $g(\tau)$ \citep{wallinga2007generation}:
\begin{linenomath*}
\begin{equation}
\frac{1}{\Ro} = \int \exp(-r\tau) g(\tau) \, d\tau.
\label{eq:euler}
\end{equation}
\end{linenomath*}
In other words, all model assumptions come down to assumptions about the exponential growth rate $r$ and the shape of the generation-interval distribution $g(\tau)$.
For example, if a model relies on strong assumptions about the underlying observation or process model, the estimated confidence intervals associated with the exponential growth rates or parameters of the generation-interval distributions will be necessarily narrow.
Therefore, it is sufficient to consider the estimates and assumptions about the exponential growth rates and the shapes of the generation-interval distributions to understand disparate estimates of the basic reproduction number.

As most studies do not report their estimates of the exponential growth rate, we summarize model outcomes using reported (either estimated or assumed) values of the basic reproductive number \Ro, mean generation interval $\bar G$ and generation-interval dispersion $\kappa$, represented by the squared coefficient of variation (Table 1);
we re-estimate the corresponding exponential growth rates later.
Study 2 only reports their assumptions about the mean generation interval; for brevity, we assume $\kappa = 0.5$ in our analysis.
Study 6 presents \Ro estimates under 12 different scenarios regarding reporting rates (0, 0.5, 1 or 2 fold increase in reporting rate) and the shapes of the generation-interval distributions (MERS-like, SARS-like, and average);
we use their baseline scenario in our analysis to remain consistent with other studies, which do not account for changes in the reporting rate.
While estimates of \Ro and associated confidence intervals for Study 6 in Table 1 represents estimates of \Ro based on $\bar G = 8\,\mathrm{days}$, we account for the uncertainty they consider for $\bar G$ in our formal analysis.
\rev{Page 7, Line 20. Figure 1 - Caption: Did the authors test the sensitivity of the value of kappa of
Study 2 (e.g., 0 or 1) on the pooled estimate?}

While most studies report confidence or credible intervals to quantify uncertainties associated with their estimates, some rely on different measures.
In particular, study 2 reports a range of \Ro for the worst and best case scenarios, which correspond to the values of \Ro such that 95\% and 5\% of the simulated total number of cases by Janaury 18, 2020 is greater than or equal to 4000, respectively;
for brevity, we treat these intervals as a 90\% credible interval in our analysis.
Uncertainty ranges reported by Study 1 and Study 7 are assumed to be uniform ranges.

%While some studies have now been published in peer-reviewed journals \citep{riou2020pattern, zhao2020preliminary} or have been updated with better uncertainty quantification \citep{readncov2}, we strictly focus on estimates that were published between January 23--26, 2020.
%The main purpose of this study is to illustrate that estimates of \Ro from different models can be understood by comparing their underlying assumptions about the exponential growth rates and generation-interval distributions, rather than to provide more precise or accurate estimate of \Ro.

\subsection{Gamma approximation framework for linking $r$ and $\Ro$}

Here, we use the gamma approximation framework to the generation-interval distribution \citep{nishiura2009transmission, mcbryde2009early, roberts2011early, trichereau2012estimation, nishiura2015theoretical, park2019practical} to (i) characterize the amount of uncertainty present in the exponential growth rates and the shape of the generation-interval distribution and (ii) assess the degree to which these uncertainties affect the estimate of \Ro.
The gamma distribution provides a reasonable approximation for generation-interval distributions of many diseases, including Ebola, influenza, measles, and rabies.
Assuming that generation intervals follow a gamma distribution with mean generation interval $\bar G$ and generation-interval dispersion $\kappa$, represented by the squared coefficient of variation of a gamma distribution, we have
\begin{linenomath*}
\begin{equation}
\Ro = \left(1 + \kappa r \bar{G}\right)^{1/\kappa}.
\label{eq:gamma}
\end{equation}
\end{linenomath*}
This equation demonstrates that a generation-interval distribution that has a larger mean (higher $\bar{G}$) or is less variable (lower $\kappa$) will give a higher estimate of \Ro for the same value of $r$ \citep{wallinga2007generation}.

\subsection{Statistical framework}

\rev{The authors consider that all the 7 studies have the same weight in the estimation of basic
reproduction weight even though their estimations and precision vary a lot. The authors might
explain why they make this hypothesis and why they did not explore a weighted estimation
based on the confidence of the estimation of R0.}

\rev{In each chain, the authors use 500,000 sampling steps and remove every 1000th step.
Therefore, the posterior distributions of each parameter consist of 2,000 samples (500 samples
per chain with 4 independent chains). Is it sufficient for the effective sample size? Is it too much
for the burn-in steps of 500,000 and the thinning steps of 1,000?}

As most studies do not report their estimates of the exponential growth rate, we first re-estimate the exponential growth rate that correspond to their model assumptions:
Since the estimates of the basic reproductive number \Ro entirely depends on the exponential growth rate and the shape of generation-interval distributions, we can calulate the exponential growth rate from the basic reproductive number \Ro, the mean generation interval $\bar G$, and the generation-interval dispersion $\kappa$.
First, to account for uncertainties in these parameters, we model reported values of the basic reproductive number \Ro, the mean generation interval $\bar G$, and the generation-interval dispersion $\kappa$ with appropriate probability distributions;
we used gamma distributions to model values reported with confidence/credible intervals and uniform distributions to model values reported with ranges (Table 2).
For example, Study 3 estimated $\Ro = 2.92$ (95\% CI: 2.28--3.67);
we model this estimate as a gamma distribution with a mean of 2.92 and a shape parameter of 67, which has a 95\% probability of containing a value between 2.28 and 3.67 (see Table 2 for a complete description).

For each study $i$, we construct a family of parameter sets by drawing 100,000 random samples from the probability distributions (Table 2) that represent the estimates of \Rx{0i} and the assumed values of $\bar G_i$ and $\kappa_i$ and calculate the exponential growth rate $r_i$ via the inverse of \eref{gamma}:
\begin{linenomath*}
\begin{equation}
r_i = \frac{\Rx{0i}^{\kappa_i} - 1}{\kappa_i \bar{G}_i}.
\end{equation}
\end{linenomath*}
This allows us to approximate the probability distributions of the estimated exponential growth rates by each study;
uncertainties in the probability distributions that we calculate for the estimated exponential growth rates will reflect the methods and assumptions that the studies rely on.

\newcommand{\gammdist}{\mathrm{Gamma}}
\begin{table}[t]
\begin{center}
\scriptsize
\begin{tabular}{l|p{4.5cm}|p{2.5cm}|p{2.7cm}}
 & Basic reproductive\newline number \Ro & Mean generation\newline interval $\bar G$ (days) & Generation-interval\newline dispersion $\kappa$ \\
\hline
Study 1 & $\mathrm{Uniform}(1.5, 3.5)$ & 10 & 1 \\
\hline
Study 2 & $\gammdist(\mathrm{mean}=2.6, \alpha=28)$ & 8.4 & 0.5 \\
\hline
Study 3 & $\gammdist(\mathrm{mean}=2.92, \alpha=67)$ & 8.4 & 0.2 \\
\hline
Study 4 & $\gammdist(\mathrm{mean}=3.8, \alpha=1400)$ & 7.6 & 0.5 \\
\hline
Study 5 & $\gammdist(\mathrm{mean}=2.2, \alpha=12)$ & $\mathrm{Uniform}(7, 14)$ & 0.5\\
\hline
Study 6 & $\gammdist(\mathrm{mean}=5.47, \alpha=54)$ & $\mathrm{Uniform}(7.6, 8.4)^\ast$ & 0.2\\
\hline
Study 7 & $\exp(r \bar G)^\dagger$ & $\mathrm{Uniform}(6, 10)$ & 0\\
\hline
\end{tabular}
\end{center}
\caption{
\textbf{Probability distributions for \Ro, $\bar G$, and $\kappa$.}
We use these probability distributions to obtain a probability distribution for the exponential growth rate $r$.
The gamma distribution is parameterized by its mean and shape $\alpha$.
Constant values are fixed according to Table 1.
$^\ast$We do not account for this uncertainty during our recalculation of the exponential growth rate $r$ because the reported estimate of $\mathcal R_0$ and its uncertainty assumes $\bar G = 8$; the original article reports three $\mathcal R_0$ (and 95\% CIs) estimates using three different values of $\bar G$: 7.6 (MERS-like), 8 (average), and 8.4 (SARS-like).
We still account for this uncertainty in our pooled estimates ($\mu_G$).
$^\dagger$Study 6 uses the IDEA model \citep{fisman2013idea}, through which the authors effectively fit an exponential curve to the cumulative number of confirmed cases without propagating any statistical uncertainty.
Instead of modeling \Ro with a probability distribution and recalculating $r$, we use $r=0.114\,\mathrm{days}^{-1}$, which explains all uncertainty in the reported \Ro, when combined with the considered range of $\bar G$.
}
\end{table}

We construct pooled estimates for each parameter ($r$, $\bar G$, and $\kappa$) using a Bayesian multilevel modeling approach, which assumes that the parameters across different studies come from the same gamma distribution.
The pooled estimates, which are represented as probability distributions rather than point estimates, allow us to average across different modeling approaches, while accounting for the uncertainties in the assumptions they make:
\begin{linenomath*}
\begin{equation}
\begin{aligned}
r_i &\sim \gammdist(\mathrm{mean}=\mu_r, \mathrm{shape}=\mu_r^2/\sigma_r^2),\\
\bar{G}_i &\sim \gammdist(\mathrm{mean}=\mu_G, \mathrm{shape}=\mu_G^2/\sigma_G^2),\\
\kappa_i &\sim \gammdist(\mathrm{mean}=\mu_\kappa, \mathrm{shape}=\mu_\kappa^2/\sigma_\kappa^2),\\
\end{aligned}
\end{equation}
\end{linenomath*}
where $\mu_r$, $\mu_G$, $\mu_\kappa$ represent the pooled estimates, and $\sigma_r$, $\sigma_G$, and $\sigma_\kappa$ represent between-study standard deviations.
We account for uncertainties associated with $r_i$, $\bar G_i$ and $\kappa_i$ (and their correlations), by drawing a random set from the family of parameter sets for each study at each Metropolis-Hastings step.
Since the gamma distribution does not allow zeros, we use $\kappa=0.02$ instead for Study 7.
We note that this approach does not account for non-independence between the parameter estimates made by different modelers.
As we add more models, the pooled estimates can become sharper even when the models no longer add more information.
Thus, the pooled estimator should be interpreted with care.
\rev{Authors state that “We note that this approach does not account for non- independence between the parameter estimates made by different modelers.” Authors should discuss this point more in depth. This seems to me an important limitation. With this respect I wonder what data the studies focus on. Are all analyzing the same epidemic (in China) considering same or similar data? Again, more details on these studies (in this case the location, period of the epidemic, and data source) should be provided.}

\rev{Page 6, Line 29. Are the priors for $\mu_r$, $\mu_G$, and $\mu_\kappa$ weakly informative? Did the
authors check the sensibility with new priors?}
We use weakly informative priors on hyperparameters:
\begin{linenomath*}
\begin{equation}
\begin{aligned}
\mu_r &\sim \gammdist(\mathrm{mean}=1/7\,\mathrm{days}^{-1},\,\mathrm{shape}=2)\\
\mu_G &\sim \gammdist(\mathrm{mean}=7\,\mathrm{days},\,\mathrm{shape}=2)\\
\mu_\kappa &\sim \gammdist(\mathrm{mean}=0.5,\,\mathrm{shape}=2)\\
(\sigma_r, \sigma_G, \sigma_\kappa) &\sim \textrm{half-normal}(0, 10).
\end{aligned}
\end{equation}
\end{linenomath*}
We followed recommendations outlined in \cite{gelman2006prior}, parameterizing the top-level gamma distributions in terms of their means and standard deviations and imposing weakly informative prior distributions on between-study standard deviations, i.e., $\textrm{half-normal}(0, 10)$.
We had initially used gamma priors with small shape parameters ($< 1$) on between-study shape parameters ($=\mu^2/\sigma^2$) but found this put too much prior probability on large between-study variances. This phenomenon is a known problem \citep{gelman2006prior}.
Alternative choices of prior for the between-study shape parameters are also suboptimal: imposing strong priors (e.g. half-$t(\mu=0,\sigma=1,\nu=4)$  assumes \textit{a priori} that between-study variance is large,  while weak priors (e.g. half-Cauchy(0,5)) can lead to poor mixing.
\rev{Page 6 Lines 44-47. Please clarify the sentence: “Alternative choices of prior … can lead to
poor mixing”.}

We run 4 independent Markov Chain Monte Carlo chains each consisting of 500,000 burnin steps and 500,000 sampling steps.
Posterior samples are thinned every 1000 steps.
Convergence is assessed by ensuring that the Gelman-Rubin statistic is below 1.01 for all hyperparameters \citep{gelman1992inference};
trace plots and marginal posterior distribution plots are presented in Appendix.
95\% credible intervals (CI) are calculated by taking 2.5\% and 97.5\% quantiles from the marginal posterior distribution for each parameter.

\section{Results}

\rev{In Figures 2 and 3, the authors replace the reported parameters (r, $\bar G$, and kappa) with the
pooled estimates mu ($\mu_r$, $\mu_G$, and $\mu_\kappa$). The authors do not give information about
the other sources of uncertainty linked with the variation of parameter sigma ($\sigma_r$, $\sigma_G$,
$\sigma_\kappa$).}

\rev{In figure 1, the posterior density distribution for the 3 parameters [exponential growth rate (r),
Mean generation interval ($\bar G$), Generation interval dispersion (kappa))] must replace the
segment representing the median and the CI. In figure 1, the legend must be self-sufficient. ‘See
text’ it is not possible. A short explanation is necessary.}

\rev{In the first paragraph of the Results page 7 the value of the median and the CI of the 3
parameters must be presented.}

\fref{assumption} compares the reported values of the exponential growth rate $r$, mean generation interval $\bar G$, and the generation-interval dispersion $\kappa$ from different studies with the pooled estimates that we calculate from our multilevel model.
We find that there is a large uncertainty associated with the underlying parameters;
many models rely on stronger assumptions that ignore these uncertainties.
Surprisingly, no studies take into account how generation-interval dispersion affects their estimates of \Ro:
all studies assumed fixed values for $\kappa$, ranging from 0 to 1.
Assuming fixed parameter values can lead to overly strong conclusions \citep{elderd2006uncertainty}.

\begin{figure}[t]
\includegraphics[width=\textwidth]{compare_assumption.pdf}
\caption{
\textbf{Comparisons of the reported parameter values with our pooled estimates.}
We inferred point estimates (black), uniform distributions (orange) or confidence/credible intervals (purple) for each parameter from each study, and combined them into pooled estimates (red; see text).
Open triangle: we assumed $\kappa=0.5$ for Study 2 which does not report generation-interval dispersion.
}
\label{fig:assumption}
\end{figure}

\rev{The estimation of R0 median and its credible interval “2.9 (95\% CI: 2.1-4.5)” must be presented
in page 7 with the figure 2 in the second paragraph of Results and not at the end of the Results
part (page 9 line 26).}

\rev{The uncertainty related to kappa is not described in Figure 2. Even the authors mentioned in
Page 7, Lines 44-50, the results should also be visualized in Figure 2 along with other
parameters.}

\rev{Page 7, Line 38. The authors might give some information about the incubation period of
COVID-19. It could explain why using the data of 26 January is still acceptable even if it was the
3rd day of the city lockdown.}
\fref{eff} shows how propagating uncertainty in different combinations would affect estimates and CIs for \Ro. For illustrative purposes, we use our pooled estimates, which may represent a reasonable proxy for the state of knowledge as of January 23--26 (\fref{eff}A).
Comparing the models that include only some sources of uncertainty to the ``all'' model, we see that propagating error from the growth rate (which all but one of the studies reviewed did) is absolutely crucial: the middle bar (``GI mean''), which lacks growth-rate uncertainty, is relatively narrow.
In this case, propagating error from the mean generation interval has negligible effect compared to propagating the uncertainty in $r$.
Uncertainty in the generation-interval dispersion $\kappa$ also has important effects as it determines the functional form of the relationship between $r$ and \Ro (compare ``growth rate + GI mean'' with ``all'').
For example, reducing $\kappa$ from 1 (assuming exponentially distributed generation intervals) to 0 (assuming fixed generation intervals) changes the $r$--\Ro\ relationship from linear to exponential, therefore increasing the sensitivity of \Ro estimates to $r$ and $\bar G$.

As uncertainty associated with the exponential growth rate decreases, accounting for uncertainties in generation intervals becomes even more important (\fref{eff}B).
Propagating error only from the growth rate gives very narrow credible intervals in this case. 
Likewise, propagating errors from the growth rate and the mean generation interval gives wider but still too narrow credible intervals.
We expect this hypothetical example to better reflect more recent scenarios, as increased data availability will allow researchers to estimate $r$ with more certainty.

\rev{Please clearly explain the formula in the caption of Figure 2 (Page 8, line 33) in the Methods
section.}

\begin{figure}[!ht]
\includegraphics[width=\textwidth]{figure2.pdf}
\caption{
  \textbf{Effects of $r$, $\bar G$, and $\kappa$ on the estimates of \Ro.}
We compare estimates of \Ro under five scenarios that propagate different combinations of uncertainties (A) based on our pooled estimates ($\mu_r$, $\mu_G$, and $\mu_\kappa$) and (B) assuming a 4-fold reduction in uncertainty of our pooled estimate of the exponential growth rate (using $(\mu_r + 3\times\mathrm{median}(\mu_r))/4$, instead).
\textbf{base}: \Ro estimates based on the median estimates of $\mu_r$, $\mu_G$, and $\mu_\kappa$.
\textbf{growth rate}: \Ro estimates based on the the posterior distribution of $\mu_r$ while using median estimates of $\mu_G$ and $\mu_\kappa$.
\textbf{GI mean}: \Ro estimates based on the the posterior distribution of $\mu_G$ while using median estimates of $\mu_r$ and $\mu_\kappa$.
\textbf{growth rate + GI mean}: \Ro estimates based on the the joint posterior distributions of $\mu_r$ and $\mu_G$ while using a median estimate of $\mu_\kappa$.
\textbf{all}: \Ro estimates based on the joint posterior distributions of  $\mu_r$, $\mu_G$, and $\mu_\kappa$.
Vertical lines represent the 95\% credible intervals.
}
\label{fig:eff}
\end{figure}

\rev{Page 8, Line 50. Please clarify the sentence: ”We find that incorporating uncertainties one at a
time increases the width of the confidence intervals in all but 7 cases”. It is not clear particularly
at “in all but 7 cases”.}
We also compare the estimates of \Ro across different studies by 
replacing their values of $r$, $\bar G$, and $\kappa$ with our pooled estimates ($\mu_r$, $\mu_G$, and $\mu_\kappa$, respectively) one at a time and recalculating the basic reproductive number \Ro (\fref{R0}).
This procedure allows us to assess the sensitivity of the estimates of \Ro across appropriate ranges of uncertainties.
We find that incorporating uncertainties one at a time increases the width of the credible intervals in all but 7 cases.
We estimate narrower credible intervals for Study 3, Study 6, and Study 7 when we account for proper uncertainties in the generation-interval dispersion because they assume a narrow generation-interval distribution (compare ``base'' with ``generation-interval dispersion'');
when higher values of $\kappa$ are used, their estimates of \Ro become less sensitive to the values of $r$ and $\bar G$, giving narrower credible intervals.
We estimate narrower credible intervals for Study 5 and Study 7 when we account for proper uncertainties in the mean generation interval (compare ``base'' with ``mean generation interval'') because the range of uncertainty in the mean generation interval $\bar G$ they consider is much wider than the pooled range (\fref{assumption}).
Substituting the reported $r$ or $\bar G$ from Study 1 with our pooled estimates give narrower credible intervals for similar reasons.

We find that accounting for uncertainties in the estimate of $r$ has the largest effect on the estimates of \Ro\ in most cases (\fref{R0}).
For example, recalculating \Ro for Study 7 by using our pooled estimate of $r$ gives $\Ro = 3.9$ (95\% CI: 2.3--8.6), which is much wider than the uncertainty range they reported (2.0--3.1).
There are two explanations for this result.
First, even though the exponential growth rate $r$ and the mean generation interval $\bar G$ have identical mathematical effects on \Ro in our framework (\eref{gamma} in Methods),
$r$ is more influential in this case because it is associated with more uncertainty (\fref{assumption}).
Second, assuming a fixed generation interval ($\kappa=0$) makes the estimate of \Ro too sensitive to $r$ and $\bar G$.
One exception is Study 1: we find this estimate of \Ro is most sensitive to generation-interval dispersion $\kappa$.
This is because Study 1 assumes an exponentially distributed generation interval ($\kappa=1$): estimates that rely on this assumption make \Ro relatively insensitive and thus tend to have particularly narrow credible intervals.

\begin{figure}[!th]
\includegraphics[width=\textwidth]{compare_R0.pdf}
\caption{
\textbf{Sensitivity of the reported \Ro estimates with respect to our pooled estimates of the underlying parameters.}
We replace the reported parameter values (growth rate $r$, mean generation interval $\bar G$, and generation-interval dispersion $\kappa$) with our corresponding pooled estimates ($\mu_r$, $\mu_G$, and $\mu_\kappa$) one at a time and recalculate \Ro (\textbf{growth rate}, \textbf{mean generation interval}, and \textbf{generation-interval dispersion}).
The pooled estimate of \Ro is calculated from the joint posterior distribution of $\mu_r$, $\mu_G$, and $\mu_\kappa$ (\textbf{all});
this corresponds to replacing all reported parameter values with our pooled estimates, which gives identical results across all studies.
Horizontal dashed lines represent the 95\% credible intervals of our pooled estimate of \Ro.
The reported \Ro estimates (\textbf{base}) have been adjusted to show the approximate 95\% credible interval using the probability distributions that we defined if they had relied on different measures for parameter uncertainties.
}
\label{fig:R0}
\end{figure}

Finally, we incorporate all uncertainties by using posterior samples for $\mu_r$, $\mu_G$, and $\mu_\kappa$ to recalculate \Ro and compare it with the reported \Ro estimates.
Our estimated \Ro from the pooled distribution has a median of 2.9 (95\% CI: 2.1--4.5).
While the point estimate of \Ro is similar to other reported values from this date range, the credible intervals are wider than all but one study.
This result does not imply that assumptions based on the pooled estimate are too weak;
we believe that this credible interval more accurately reflects the level of uncertainties present in the information that was available when these models were fitted.
In fact, because the pooled estimate does not account for overlap in data sources used by the models, we feel that it is more likely to be over-confident than under-confident.
Our median estimate averages over the various studies, and therefore particular studies have higher or lower median estimates.
We note in particular that, while the baseline example we used from Study 6 may appear to be an outlier, the authors of this study also explore different scenarios involving changes in reporting rate over time, under which their estimates of \Ro are similar to other reported estimates.
Here, our focus is on estimating uncertainty, not on identifying potential explanations for these discrepancies.

\section{Discussion}

\rev{In the Results section, the authors occasionally discuss the results, i.e., Page 8, line 50 to Page
9, line 23. The opinions should be in the Discussion section.
In the Discussion section, the authors should discuss the quality of the reported estimates from
7 studies used in this subject. Some of them do not provide detailed methods and assumptions.
Also, most of them are preprint articles. It could be another important source of uncertainty on
R0.}

\rev{Although the authors are in the case of the COVID-19 epidemic in Hubei before any
confinement, the authors do not mention in the introduction or in the discussion the other
sources of uncertainty linked to the data (such as the variations in diagnostic results related to
the day of the week, a saturation of diagnostic test capacity, transparency of data,
representativeness of samples and improvement of detection capacity as time goes by.}

\rev{Page 4, Table 1. Please check Study 4, the article has been revised since 27 January 2020,
and their estimates also changed.}

\rev{Page 6, Table 2. Please recheck the parameters of Study 5, which is inconsistent with the 90%
confidence interval in Table 1. The 90\% confidence interval for a Gamma distribution with mean
2.2 and shape 12 is 1.27 to 3.34, while the interval in Table 1 is 1.4 to 3.8.}

Estimating the basic reproductive number \Ro is crucial for predicting the course of an outbreak and planning intervention strategies.
Here, we use a gamma approximation \citep{park2019practical} to decompose \Ro estimates into three key quantities ($r$, $\bar G$, and $\kappa$) and apply a multilevel Bayesian framework to compare estimates of \Ro for the novel coronavirus outbreak.
Our results demonstrate the importance of accounting for uncertainties associated with the underlying generation-interval distributions, including uncertainties in the amount of dispersion in the generation intervals.
Our analysis of individual studies shows that many early estimates of \Ro rely on strong assumptions.

Of the seven studies that we reviewed, two of them directly fit their models to cumulative number of confirmed cases.
This approach can be appealing because of its simplicity and apparent robustness, but fitting a model to cumulative incidence instead of raw incidence can both bias parameters and give overly narrow confidence intervals, if the resulting non-independent error structures are not taken into account \citep{ma2014estimating, king2015avoidable}.
Naive fits to cumulative incidence data should therefore be avoided.
\rev{
At the beginning of the Discussion section authors list a series of important issues and possible sources of errors that often limit the utility of early outbreak analyses (e.g. fitting the cumulative curve of cases, assuming a fixed generation interval when in fact it varies during the outbreak, changing reporting rate, etc.). Compared with these issues the main focus of the paper (not properly accounting for the uncertainty in the generation time and growth rate parameters) appears a small caveat. Maybe author could give a sense of the importance of the element they are focusing on with the respect to the other issues?
}

Many sources of noise affect real-world incidence data, including both dynamical, or ``process'', noise (randomness that directly or indirectly affects disease transmission); and observation noise (randomness underlying how many of the true cases are reported).  
Disease modelers face the choice of incorporating one or both of these in their data-fitting and modeling steps. 
This is not always a serious problem, particularly if the goal is inferring parameters rather than directly making forecasts \citep{ma2014estimating}.
Modelers should however be aware of the possibility that ignoring one kind of error can give overly narrow confidence intervals \citep{king2015avoidable,taylor2016stochasticity}.

There are other important phenomena not covered by our simple framework. 
Examples that seem relevant to this outbreak include: changing reporting rates, reporting delays (including the effects of weekends and holidays), and changing generation intervals.
For emerging pathogens such as SARS-CoV-2, there may be an early period of time when the reporting rate is very low due to limited awareness or diagnostic resources;
for example, \cite{zhaoncov} (Study 6) demonstrated that estimates of \Ro can change from 5.47 (95\% CI: 4.16--7.10) to 3.30 (95\% CI: 2.73--3.96) when they assume 2-fold changes in the reporting rate between January 17, when the official diagnostic guidelines were released \citep{who17protocol}, and January 20.
Delays between key epidemiological timings (e.g., infection, symptom onset, and detection) can also shift the shape of an observed epidemic curve and, therefore, affect parameter estimates as well as predictions of the course of an outbreak \citep{tariq2019assessing}.
Even though a constant delay between infection and detection may not affect the estimate of the growth rate, it can still affect the associated credible intervals.
Finally, generation intervals can become shorter throughout an epidemic as intervention strategies, such as quarantine, can reduce the infectious period \citep{hethcote2002effects}.
Accounting for these factors is crucial for making accurate inferences.

Here, we focused on the estimates of \Ro that were published within a very short time frame (January 23--26).
During early phases of an outbreak, it is reasonable to assume that the epidemic grows exponentially \citep{anderson1991infectious}. However, as the number of susceptible individuals decreases, the epidemic will saturate, and estimates of $r$ used for \Ro should account for the possibility that $r$ is decreasing through time.
Although our analysis only reflects a snapshot of a fast-moving epidemic, we expect certain lessons to hold: credible intervals must combine different sources of uncertainty. 
In fact, as epidemics progress and more data becomes available, it is likely that inferences about exponential growth rate (and other epidemiological parameters) will become more precise; thus the risk of over-confidence when uncertainty about the generation-interval distribution is neglected will become greater.

We strongly emphasize the value of attention to accurate characterization of the transmission chains via contact tracing and better statistical frameworks for inferring generation-interval distributions from such data \citep{britton2019estimation}.
A combined effort between public-health workers and modelers in this direction will be crucial for predicting the course of an epidemic and controlling it.
We also emphasize the value of transparency from modelers.
Model estimates during an outbreak, even in pre-prints, should include code links and complete explanations.
We suggest using methods based on open-source tools allow for maximal reproducibility.

In summary, we have provided a basis for comparing exponential-growth based estimates of \Ro and its associated uncertainty in terms of three components: the exponential growth rate, mean generation interval, and generation interval dispersion. 
We hope this framework will help researchers understand and reconcile disparate estimates of disease transmission early in an epidemic.

\pagebreak

\section*{Funding}

BMB and DJDE were supported by Natural Sciences and Engineering Research Council (NSERC). ML was supported by Canadian Institutes of Health Research (CIHR). The funders had no role in study design, data collection and analysis, decision to publish, or preparation of the manuscript.

\section*{Competing interests}

We declare no competing interests.

\section*{Acknowledgements}

We thank Daihai He for providing helpful comments on the manuscript.

\section*{Contribution}

SWP and JD developed the statistical framework. 
SWP reviewed the published literature.
SWP performed the analysis. 
SWP, BMB, and JD created the figures. 
SWP and JD wrote the first draft.
All authors contributed to the writing and approval of the final report.

\section*{Data availability}

\texttt{R} code is available in GitHub (\url{https://github.com/parksw3/nCoV_framework}).


\pagebreak

\bibliography{ncov_abbr}

\pagebreak
\appendix
\renewcommand\thefigure{A\arabic{figure}}
\setcounter{figure}{0}    
\section*{Appendix}

\begin{figure}[!h]
\includegraphics[width=\textwidth]{posterior_chain.pdf}
\caption{
\textbf{Trace plots of the multilevel model.}
Each chain is represented by a different color.
}
\end{figure}

\pagebreak

\begin{figure}[!h]
\includegraphics[width=\textwidth]{posterior_dist.pdf}
\caption{
\textbf{Marginal posterior distributions of the multilevel model.}
Each chain is represented by a different color.
}
\end{figure}

\end{document}
