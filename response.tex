\documentclass[12pt]{article}
\usepackage[utf8]{inputenc}

\usepackage{color}

\usepackage{xspace}

\usepackage{lmodern}
\usepackage{amssymb,amsmath}

\usepackage[pdfencoding=auto, psdextra]{hyperref}

\usepackage{natbib}
\bibliographystyle{chicago}

\newcommand{\rR}{\mbox{$r$--$\cal R$}}
\newcommand{\RR}{\ensuremath{{\cal R}}}
\newcommand{\RRhat}{\ensuremath{{\hat \cal R}}}
\newcommand{\Rx}[1]{\ensuremath{{\cal R}_{#1}}} 
\newcommand{\Ro}{\ensuremath{{\mathcal R}_{0}}\xspace}
\newcommand{\Rpool}{\ensuremath{{\mathcal R}_{\textrm{\tiny{pool}}}}\xspace}
\newcommand{\Reff}{\Rx{\mathit{eff}}}
\newcommand{\Tc}{\ensuremath{C}}

\newcommand{\rev}{\subsection*}
\newcommand{\revtext}{\textsf}
\setlength{\parskip}{\baselineskip}
\setlength{\parindent}{0em}

\newcommand{\comment}[3]{\textcolor{#1}{\textbf{[#2: }\textsl{#3}\textbf{]}}}
\newcommand{\jd}[1]{\comment{cyan}{JD}{#1}}
\newcommand{\swp}[1]{\comment{magenta}{SWP}{#1}}
\newcommand{\dc}[1]{\comment{blue}{DC}{#1}}
\newcommand{\jsw}[1]{\comment{green}{JSW}{#1}}
\newcommand{\hotcomment}[1]{\comment{red}{HOT}{#1}}

\begin{document}

\noindent Dear Editor:

Thank you for the chance to revise and resubmit our manuscript. 
We have re-written a considerable amount of the manuscript in order to convince a more general readership (suggested by reviewer 1) and to provide a clearer explanation of the methods (suggested by reviewer 2).
A large proportion of the Results section has been moved to the Discussion section, as suggested by reviewer 2.
We have also included a new paragraph in the Discussion section to compare early estimates with more recent estimates of the basic reproduction number.
The main figures and the overall conclusions remain unchanged.
Below please find our responses to reviewers.

\rev{Reviewer \#1}

\revtext{
The manuscript presents a statistical framework for comparing different estimates of R0 decomposed in their building blocks (exponential growth, average generation interval, and dispersion in the generation interval). The analysis allows for pulled estimates of the different quantities, combining information from the different studies. In addition, authors provide a reflection on how the uncertainty in the different components is propagated in the R0 estimate and its confidence interval.}

\revtext{It seems to me that the paper is well written.}

Thank you!

\revtext{As a disclaimer, I must say that I am not sufficiently expert of the methodology used here to be fully confident in my judgment of the soundness and the completeness of the analysis. Maybe it is related to my limitation, but I find the main focus of the paper a little too technical for the Journal (e.g. see the last point below). Thus, I believe that authors could make an extra effort to convince a more general and interdisciplinary public on the importance of the point they are raising.}

We have made significant revisions to our manuscript to make it approachable for a wider audience.

\revtext{I am not sure whether authors incorporate well all assumptions of the studies they consider.} 

This is a key point. During the exponential growth period, the basic reproduction number entirely depends on the exponential growth rate and the generation-interval distributions.
Therefore, the effect of model assumptions on the inferred reproductive number can in fact be considered via their effects on exponential growth rate and the shape of the generation-interval distribution.
We have tried to make this point clearer in the main text:

``Despite a wide range of models considered across Study 1--7, all of them assume that the epidemic initially grows exponentially.
The IDEA model (used in Study 7) includes a discount parameter $d$ that allows the model to deviate from exponential growth when $d \neq 0$ \citep{fisman2013idea}, but Study 7 estimates $d=0$ across all parameters they consider.
Even though some studies consider reported cases up to January 26, 2020 --- three days after the travel restriction that took place on January 23, 2020 \citep{Tianeabb6105} --- the exponential growth assumption can still describe the number of reported cases reasonably well;
given the incubation period of around 5 days \citep{lauer2020incubation} as well as reporting delays of around 5 days \citep{sun2020early}, the majority of reported cases during the study periods are likely to have been infected prior to the travel ban.

When the epidemic is growing exponentially, the estimated basic reproductive number depends entirely on the exponential growth rate $r$ and the intrinsic generation-interval distribution $g(\tau)$, which describes the infection time of secondary cases caused by a primary case in a fully susceptible population \citep{champredon2015intrinsic}, via the Euler-Lotka equation \citep{wallinga2007generation}:
\begin{equation}
\frac{1}{\Ro} = \int \exp(-r\tau) g(\tau) \, d\tau.
\label{eq:euler}
\end{equation}
In other words, all model assumptions reduce to assumptions about the exponential growth rate $r$ and the shape of the generation-interval distribution $g(\tau)$.
If a model relies on strong assumptions about the underlying observation or process model, the estimated confidence or credible intervals associated with the exponential growth rates or parameters of the generation-interval distributions will necessarily be narrow.
Therefore, it is sufficient to consider the estimates and assumptions about the exponential growth rates and the shapes of the generation-interval distributions to understand disparate estimates of the basic reproductive number.''

\revtext{In addition, it is not clear to me on what extent the assumptions made by authors themselves affect the results - e.g. the assumption of a gamma and uniform distribution for the estimates adopted to compute the growth rate.}

Since we are interested in the uncertainty, expressed by probability distributions, our analysis is consistent with the uncertainty reported by the authors. We have added following sentences in the main text:

``This approach of re-estimating the exponential growth rate does not affect the uncertainty captured by our analysis because we are re-estimating the probability distribution of $r_i$ that is consistent with the reported values of $(\Ro)_i$, $\bar G_i$, and $\kappa_i$;
in other words, we still obtain the same degree of associated uncertainty in $(\Ro)_i$ if we calculate it from $r_i$, $\bar G_i$, and $\kappa_i$.''

\revtext{Other hypotheses underlie these works other than average generation interval, its coefficient of variation and the exponential growth. For instance, what about the offspring distribution (Poisson vs. negative binomial), or the observation model? Authors should provide a synthesis of all assumptions of these studies}

See above. These assumptions will be reflected in the uncertainty distributions reported by the authors, which we then incorporate into our analysis. We have added a new table to provide a better overview of the models and data used in each study.

\revtext{and discuss how well the gamma approximation framework embrace all cases.}

The gamma approximation framework assumes that the generation-interval distribution is gamma distributed.
Other assumptions mentioned above (e.g., offspring distributions or the observation model) do not affect the generation-interval distributions.
Moreover, this approximation has been widely used previously and has been tested across multiple diseases.
We have clarified the text and included appropriate citations:

``Here, we use the gamma approximation framework to the generation-interval distribution \citep{nishiura2009transmission, mcbryde2009early, roberts2011early, trichereau2012estimation, nishiura2015theoretical, park2019practical} to (i) characterize the amount of uncertainty present in the exponential growth rates and the shape of the generation-interval distribution and (ii) assess the degree to which these uncertainties affect the estimate of \Ro.
The gamma distribution provides a reasonable approximation for generation-interval distributions of many diseases, including Ebola, measles, and rabies \citep{park2019practical}.
Studies 1, 5, 6, and 7 also used a gamma distribution (including the special cases of Dirac delta and exponential distributions) to model the generation-interval distribution for SARS-CoV-2.''

\revtext{
Authors state that “We note that this approach does not account for non- independence between the parameter estimates made by different modelers.” Authors should discuss this point more in depth. This seems to me an important limitation. With this respect I wonder what data the studies focus on. Are all analyzing the same epidemic (in China) considering same or similar data? Again, more details on these studies (in this case the location, period of the epidemic, and data source) should be provided.}

We now provide a table that summarizes this information. We have also added the following text:

``Our approach does not account for non-independence between the parameter estimates made by different modelers.
In this case, most estimates primarily depend on reported cases from China, particularly from Wuhan City.
Differences among estimates are primarily driven by differences in estimation methods and underlying assumptions, rather than by epidemiological differences.
The pooled estimates can become sharper (i.e., have narrower credible intervals) as we add more models even when the models or the data no longer add more information about the epidemic.
Since SARS-CoV-2 had primarily spread in Wuhan City, China during this period, it was not possible to include independent sources of data from other countries.
Thus, the pooled estimator should be interpreted with care.''

\revtext{
At the beginning of the Discussion section authors list a series of important issues and possible sources of errors that often limit the utility of early outbreak analyses (e.g. fitting the cumulative curve of cases, assuming a fixed generation interval when in fact it varies during the outbreak, changing reporting rate, etc.). Compared with these issues the main focus of the paper (not properly accounting for the uncertainty in the generation time and growth rate parameters) appears a small caveat. Maybe author could give a sense of the importance of the element they are focusing on with the respect to the other issues?}

All of these issues raised here directly affect estimates of growth rate $r$, which in turn affect estimates of the basic reproduction number \Ro.
Therefore, comparing the growth rate with generation-interval distribution parameters allow us to take these assumptions into account implicitly. We have made this clearer in the main text.

\rev{Reviewer \#2}

\revtext{The subject is interesting for the ongoing pandemic. The authors propose an estimation method
for the basic reproduction number and describe its uncertainty from the estimates made by
several early studies. The consensus epidemiological estimate could be beneficial for making
control strategies.}

Thank you!

\revtext{The authors propose a statistical method for an epidemiological situation. They combine basic
reproduction rate estimates and other estimates in the early stage of the outbreak and their
uncertainty to obtain a global estimate with its uncertainty. So, this subject is at the interface of
biomathematics and epidemiology of the infectious disease, which is the scope of the journal.}

\revtext{The interesting and primordial subject of the estimation of R0 and its uncertainty is studied using
the Bayesian inference tools. The authors do not develop enough the underlying hypotheses
and the method used. Many elements are not described, even in the appendix. This should be
described to estimate the validity of these underlying assumptions.}

\revtext{More specific points:}

\revtext{The authors must use the same variable names and their units throughout the documents: e.g.,
for kappa, which is a key parameter in this study. In the document, it appears in following
various terms that make the reader confused: GI variation, squared coefficient of variation,
generation-interval dispersion. Please always keep the same terms for the same variables. The
same remark is also applied to the mean generation interval.}

Done.

\revtext{The unit of time changes throughout the document. Choose the same time unit (e.g., days,
which is the most used and change the Equation 5, Page 6, which is in weeks).}

Done.

\revtext{As in the Bayesian inference, there are no Confidence Intervals like in frequentist inference, but
Credible Interval. The authors should replace Confidence Intervals with Credible Intervals.}

We refer to uncertainty intervals that we estimate as credible intervals as suggested. We still use confidence intervals to refer to uncertainty intervals reported in the original study if it relies on frequentist inference.

\revtext{Methods}

\revtext{The authors should clearly explain the hypotheses they have done in the Methods section (e.g.,
the source of the uncertainty).}

We have tried to clarify assumptions of our Bayesian multilevel model as much as possible. 
In particular, we made it clear that we do not rely on time series data or fit a model to time series data; 
we take reported estimates and average them.
We also explain potential sources of uncertainty in the reported estimates.

\revtext{The authors do not describe all the methods used in this article within the Methods section.
Some of them are in the figure captions, e.g., Figures 2 and 3. Also, the authors should describe
all the methods in this section. Using subheadings could help the reader to better understand
the methods, e.g., likelihood, prior distributions, sensitivity analysis, etc.}

We now explain all methods, including those that were described in figure and table captions, within the Methods section.
We left simplified versions of the descriptions in captions to assist readers in interpreting the tables and figures.
We have also added subheadings in the Methods section.

\revtext{Please clearly explain the formula in the caption of Figure 2 (Page 8, line 33) in the Methods
section.}

We have moved the formula to the Results section to provide a better context for improved explanation: 

``We further explore how the effects of uncertainties in generation-interval distributions change when the exponential growth rate is more certain.
This hypothetical example reflects scenarios, in which increased data availability allows researchers to estimate $r$ with more certainty.
To simulate estimates of the exponential growth rate with stronger confidence, we use $\hat{\mu}_r = (\mu_r + 3\times\mathrm{median}(\mu_r))/4$ instead of $\mu_r$ (\fref{eff}B); 
then $\hat{\mu}_r$ has the same median as $\mu_r$ but the associated 95\% CI is 4 times narrower (0.16--0.19 $\textrm{days}^{-1}$).
As uncertainty associated with the exponential growth rate decreases, accounting for uncertainties in generation intervals becomes even more important.
Propagating error only from the growth rate gives very narrow credible intervals in this case. 
Propagating errors from the mean generation interval or generation-interval dispersion gives more realistic but still narrow credible intervals.''

\revtext{The Bayesian estimation framework is not clearly explained. Readers could easily get confused
about which variables serve as observed data, parameters, or hyperparameters. It is also not
very clear about the number of observed data, e.g., i = 7 means that there are only 7 data (1 per
study) or there are 700,000 simulated data (100,000 per study) in the model. We strongly
suggest that the authors should provide a directed acyclic graph (DAG) representing the entire
model, which is a common practice for the Bayesian estimation.}

We have made significant revisions to the Methods section. 
In particular, we made it very clear that we are simply using Bayesian estimation to average across the reported estimates and are not relying on any time series data or dynamical model.
We hope that the new text is sufficiently clear; a directed acyclic graph to explain that we are doing Bayesian averaging seems like overkill.

\revtext{The authors consider that all the 7 studies have the same weight in the estimation of basic
reproduction weight even though their estimations and precision vary a lot. The authors might
explain why they make this hypothesis and why they did not explore a weighted estimation
based on the confidence of the estimation of R0.}

We used Bayesian multilevel models precisely for this reason (to weight the studies appropriately based on reported certainty.)

\revtext{In each chain, the authors use 500,000 sampling steps and remove every 1000th step.
Therefore, the posterior distributions of each parameter consist of 2,000 samples (500 samples
per chain with 4 independent chains). Is it sufficient for the effective sample size? Is it too much
for the burn-in steps of 500,000 and the thinning steps of 1,000?}

This may be an overkill but we wanted to make sure that the posterior samples are not autocorrelated.
On the other hand, this is definitely sufficient for the effective sample size to reliably calculate the 95\% confidence intervals, especially given that we're only averaging across previously estimated parameter values, rather than explicitly fitting a model to data. We now ensure that effective sample sizes are greater than at least 1,000 for all parameters.

\revtext{Results}

\revtext{In Figures 2 and 3, the authors replace the reported parameters (r, $\bar G$, and kappa) with the
pooled estimates mu ($\mu_r$, $\mu_G$, and $\mu_\kappa$). The authors do not give information about
the other sources of uncertainty linked with the variation of parameter sigma ($\sigma_r$, $\sigma_G$,
$\sigma_\kappa$).}

We have sampled appropriately from the Bayesian posteriors. We talk about pooled estimates of $\mu_r$, $\mu_G$, and $\mu_\kappa$ because those are used directly to calculate the reproduction number.

\revtext{The authors do not provide the validation on their estimates, e.g., a posterior predictive check
and comparisons with the reported cases during the study period.}

We do not rely on time series data or fit a model to time series data. Therefore, it is not possible to compare with the reported cases.

\revtext{In figure 1, the posterior density distribution for the 3 parameters [exponential growth rate (r),
Mean generation interval ($\bar G$), Generation interval dispersion (kappa))] must replace the
segment representing the median and the CI. In figure 1, the legend must be self-sufficient. ‘See
text’ it is not possible. A short explanation is necessary.}

We have changed the figure caption as follows:

``We inferred point estimates (black), uniform distributions (orange) or confidence/credible intervals (purple) for each parameter from each study, and combined them into pooled estimates using a Bayesian multilevel model (red).
Points represent medians calculated from the parameter set $(\bar{G}_{i}, \kappa_{i}, r_{i})$ for each study $i$ (orange and purple).
Error bars represent 95\% equi-tailed quantiles calculated from the parameter set $(\bar{G}_{i}, \kappa_{i}, r_{i})$ for each study $i$.
Red density plots represent distributions of 2000 posterior samples.
Open triangle: we assumed $\kappa=0.5$ for Study 2 which does not report generation-interval assumptions.''

\revtext{In the first paragraph of the Results page 7 the value of the median and the CI of the 3
parameters must be presented.}

Done:

``Figure 1 compares the estimated values of the exponential growth rate $r$, mean generation interval $\bar G$, and the generation-interval dispersion $\kappa$ from different studies with the pooled estimates that we calculate from our multilevel model:
$\mu_r = 0.17\,\textrm{days}^{-1}$ (95\% CI: 0.12--0.25 $\textrm{days}^{-1}$),
$\mu_G = 8.51\,\textrm{days}$ (95\% CI: 7.60--9.63 days),
and
$\mu_\kappa = 0.50$ (95\% CI: 0.26--1.10).''

\revtext{The estimation of R0 median and its credible interval “2.9 (95\% CI: 2.1-4.5)” must be presented
in page 7 with the figure 2 in the second paragraph of Results and not at the end of the Results
part (page 9 line 26).}

Done:

``Comparing the estimates that include only some sources of uncertainty to the pooled estimate ($\Rpool = 3.0$; 95\% CI: 2.1--4.6; see `all' in \fref{eff}), we see that propagating error from the growth rate (as done by all but one studies reviewed) is absolutely crucial: uncertainty in the pooled estimates for both middle bars ($\mu_G$ and $\mu_\kappa$), which lack growth-rate uncertainty, are overly narrow.''

\revtext{The uncertainty related to kappa is not described in Figure 2. Even the authors mentioned in
Page 7, Lines 44-50, the results should also be visualized in Figure 2 along with other
parameters.}

We now incorporate the uncertainty related to kappa in Figure 2.

\revtext{Discussions}

\revtext{In the Results section, the authors occasionally discuss the results, i.e., Page 8, line 50 to Page
9, line 23. The opinions should be in the Discussion section.}

We have removed these examples, and other interpretations and opinions, from the Results section.

\revtext{In the Discussion section, the authors should discuss the quality of the reported estimates from
7 studies used in this subject. Some of them do not provide detailed methods and assumptions.
Also, most of them are preprint articles. It could be another important source of uncertainty on
R0.}

This is a very good point. We have added the following paragraph to the main text:

``Here, we focused on the estimates of \Ro that were published within a very short time frame (January 23--26, 2020).
Since the estimates were published as pre-prints, rather than in peer-reviewed journals, the quality of the analyses as well as the resulting estimates were not necessarily finalized.
For example, Study 4 initially estimated $\Ro = 3.8$ (95\% CI: 3.6--4.0; \cite{readncov}) but revised their estimate on January 28, 2020 to $\Ro = 3.11$ (95\% CI: 2.39--4.13; \cite{readncov2});
we did not include their revised estimates in our analysis in order to focus on available information at the very beginning of the outbreak.
Some studies also lack detailed description of their methods, data, and/or assumptions.
The variation in quality of these analyses adds further uncertainty to their results that is not captured by their uncertainty quantification (e.g., reported credible intervals) or by our analysis.''

\revtext{Although the authors are in the case of the COVID-19 epidemic in Hubei before any
confinement, the authors do not mention in the introduction or in the discussion the other
sources of uncertainty linked to the data (such as the variations in diagnostic results related to
the day of the week, a saturation of diagnostic test capacity, transparency of data,
representativeness of samples and improvement of detection capacity as time goes by.}

Some of the issues raised here (e.g., day of the week effect and limited diagnostic resources) were already mentioned in the Discussion of the previous version of the manuscript. We now mention other points as well:

``Our simple framework neglects some other important phenomena.
Examples that seem relevant to this outbreak include: changing reporting rates; reporting delays (including the effects of weekends and holidays); and changing generation intervals.
For emerging pathogens such as SARS-CoV-2, there may be an early period of time when the reporting rate is very low due to limited awareness or diagnostic resources;
for example, \cite{zhaoncov} (Study 6) demonstrated that estimates of \Ro can change from 5.47 (95\% CI: 4.16--7.10) to 3.30 (95\% CI: 2.73--3.96) when they assume 2-fold changes in the reporting rate between January 17, when the official diagnostic guidelines were released \citep{who17protocol}, and January 20.
Delays between key epidemiological timings (e.g., infection, symptom onset, and detection) can also shift the shape of an observed epidemic curve and, therefore, affect parameter estimates as well as predictions of the course of an outbreak \citep{tariq2019assessing}.
Even though a constant delay between infection and detection may not affect the estimate of the growth rate, it can still affect the associated credible intervals.
Other factors related to reporting --- including changes in case definition, saturation in diagnostic test capacity, transparency of data, and representativeness of samples --- will also affect estimation and inference.
Finally, generation intervals can become shorter throughout an epidemic as intervention strategies such as isolation of detected cases can reduce the infectious period \citep{hethcote2002effects};
since we are primarily focusing on the outbreak in Wuhan City before confinement, generation intervals are unlikely to vary significantly.
All of these factors, including fitting to cumulative curves or ignoring process errors, affect the estimation of the exponential growth rate (as well as the associated uncertainties), which in turn affects the estimation of the basic reproductive number.''

\revtext{Page 4, Table 1. Please check Study 4, the article has been revised since 27 January 2020,
and their estimates also changed.}

We now clearly explain why we do not consider revised estimates. See above in the discussion about pre-prints. 

\revtext{Page 6, Table 2. Please recheck the parameters of Study 5, which is inconsistent with the 90\%
confidence interval in Table 1. The 90\% confidence interval for a Gamma distribution with mean
2.2 and shape 12 is 1.27 to 3.34, while the interval in Table 1 is 1.4 to 3.8.}

We match probability, not necessarily equi-tailed quantiles. A Gamma distribution distribution with mean
2.2 and shape 12 has a 90\% probability of containing a value between 1.27 and 3.34. We make this clear in our main text now:

``We used gamma distributions to model values reported with confidence or credible intervals (CI) and uniform distributions to model values reported with ranges;
when confidence or credible intervals are reported, we parameterize the gamma distribution such that (i) its mean matches the estimated value and (ii) the probability that a random variable following the specified gamma distribution falls between the lower and upper confidence or credible limits is equal to the reported confidence or credible level. 
This probability does not need to be equi-tailed.
For example, Study 3 estimated $\Ro = 2.92$ (95\% CI: 2.28--3.67);
we model this estimate as a gamma distribution with a mean of 2.92 and a shape parameter of 67, which has a 95\% probability of containing a value between 2.28 and 3.67 (see Table 2 for a complete description).''

\revtext{Page 6, Line 29. Are the priors for $\mu_r$, $\mu_G$, and $\mu_\kappa$ weakly informative? Did the
authors check the sensibility with new priors?}

We believe that these priors are sufficiently wide and therefore did not try different priors.
We also perform a prior predictive simulations on \Ro based on $\mu_r$, $\mu_G$, and $\mu_\kappa$ to provide further justification.
We provide a better justification in the main text:

``These priors are chosen such that their 95\% quantile ranges are sufficiently wider than biologically realistic parameter ranges.
Specifically, 95\% quantile ranges for $\mu_r$, $\mu_G$, and $\mu_\kappa$ are 0.02--0.40 $\mathrm{days}^{-1}$, 0.8--19.5 days, and 0.1--1.4, respectively;
95\% prior quantile range for \Ro then corresponds to 1.05--12.00.
Parameters that are outside these ranges are biologically unrealistic for SARS-CoV-2 outbreaks.
Therefore, we do not expect our results to be sensitive to these priors.''

\revtext{Page 6 Lines 44-47. Please clarify the sentence: “Alternative choices of prior … can lead to
poor mixing”.}

Done.

``Alternative choices of prior for the between-study shape parameters are also suboptimal. 
Imposing strong priors (e.g. half-$t(\mu=0,\sigma=1,\nu=4)$) assumes \textit{a priori} that between-study variance is large (and therefore does not pool different estimates sufficiently).
Overly weak priors (e.g. half-Cauchy(0,5)) leads to inefficient sampling and poor convergence.''

\revtext{Page 7, Line 20. Figure 1 - Caption: Did the authors test the sensitivity of the value of kappa of
Study 2 (e.g., 0 or 1) on the pooled estimate?}

We now check the sensitivity of the value of kappa of Study 2 on the pooled estimate of the basic reproduction number.

``However, our estimate of \Rpool is relatively insensitive to our assumption of $\kappa=0.5$ for Study 2: assuming $\kappa=0.1$ gives $\Rpool = 3.0$ (95\% CI: 2.2--4.7), whereas assuming $\kappa=0.9$ gives $\Rpool = 2.9$ (95\% CI: 2.1--4.4).''

\revtext{Page 7, Line 38. The authors might give some information about the incubation period of
COVID-19. It could explain why using the data of 26 January is still acceptable even if it was the
3rd day of the city lockdown.}

Done.

``Even though some studies consider reported cases up to January 26, 2020 --- three days after the travel restriction that took place on January 23, 2020 \citep{Tianeabb6105} --- the exponential growth assumption can still describe the number of reported cases reasonably well;
given the incubation period of around 5 days \citep{lauer2020incubation} as well as reporting delays of around 5 days \citep{sun2020early}, the majority of reported cases during the study periods are likely to have been infected prior to the travel ban.''

\revtext{Page 8, Line 50. Please clarify the sentence: ”We find that incorporating uncertainties one at a
time increases the width of the confidence intervals in all but 7 cases”. It is not clear particularly
at “in all but 7 cases”.}

We have changed this explanation (and added Sensitivity Analysis section in the Methods): 

``Finally, Figure 3 compares the base estimates (based on $r_i$, $\bar G_i$, and $\kappa_i$ for each study $i$) and 21 substitute estimates (3 parameter substitutions $\times$ 7 studies).
All but 8 substitute estimates have wider credible intervals compared to their corresponding base estimates --- the cases with more certain substitute estimates are the $\bar G$-substitute estimates for Study 1 and 7, $r$-substitute estimates for Study 1 and 2, and  $\kappa$-substitute estimates for Study 3, 6, and 7.''

\rev{Reviewer \#3}

\revtext{
Here, Park and colleagues present a straight-forward, but important analysis of 7 early estimates of R0 for the ongoing covid outbreak. They stress that multiple forms of uncertainty have to be taken into account when calculating R0 and its confidence intervals, lest the intervals be too narrow and/or the estimates over-confident. The methods are sound, and the results compelling. I really don’t have any major comments, though I wonder if it would be better suited to another journal. I do apologize to the authors for the tardiness of my review.}

As stated in the journal scope, JRSI publishes research applying mathematics and physics to the biological and medical sciences. We assume that the ``and-s'' and both sides implies that any combination is sufficiently in scope, i.e., math+bio, math+medical sciences, physics+bio, physics+medical sciences, or a combination. This paper applies mathematical methods to pressing epidemiological questions of broad relevance, both within the field of epidemiology and outside given the impacts of $\Ro$ estimation on forecasting and control efforts.

\pagebreak

\bibliography{ncov_abbr.bib}

\end{document}
