\documentclass[12pt]{article}
\usepackage[utf8]{inputenc}

\usepackage{color}

\usepackage{xspace}

\usepackage{lmodern}
\usepackage{amssymb,amsmath}

\usepackage[pdfencoding=auto, psdextra]{hyperref}

\newcommand{\rR}{\mbox{$r$--$\cal R$}}
\newcommand{\RR}{\ensuremath{{\cal R}}}
\newcommand{\RRhat}{\ensuremath{{\hat \cal R}}}
\newcommand{\Rx}[1]{\ensuremath{{\cal R}_{#1}}} 
\newcommand{\Ro}{\ensuremath{{\mathcal R}_{0}}\xspace}
\newcommand{\Rpool}{\ensuremath{{\mathcal R}_{\textrm{\tiny{pool}}}}\xspace}
\newcommand{\Reff}{\Rx{\mathit{eff}}}
\newcommand{\Tc}{\ensuremath{C}}

\newcommand{\rev}{\subsection*}
\newcommand{\revtext}{\textsf}
\setlength{\parskip}{\baselineskip}
\setlength{\parindent}{0em}

\newcommand{\comment}[3]{\textcolor{#1}{\textbf{[#2: }\textsl{#3}\textbf{]}}}
\newcommand{\jd}[1]{\comment{cyan}{JD}{#1}}
\newcommand{\swp}[1]{\comment{magenta}{SWP}{#1}}
\newcommand{\dc}[1]{\comment{blue}{DC}{#1}}
\newcommand{\jsw}[1]{\comment{green}{JSW}{#1}}
\newcommand{\hotcomment}[1]{\comment{red}{HOT}{#1}}

\begin{document}

\noindent Dear Editor:

Thank you for the chance to revise and resubmit our manuscript.
Below please find our responses to reviewers.

\rev{Reviewer \#1}

\revtext{Authors have properly addressed my comments and I believe that the manuscript is now suitable for publication}

Thank you.

\rev{Reviewer \#2}

\revtext{
The authors have made a good effort of explanation in the writing of the material and methods
part. They did not wish to carry out a DAG. Indeed, the Bayesian estimation part is relatively
simple and does not necessarily require a DAG. However, the whole process is complex and
to facilitate the understanding of the readers a conceptual framework of the whole study is
more than necessary.}

We have added the following text in the last paragraph of the introduction to provide a conceptual overview of the framework:

``Instead, we model the estimate of \Ro (as well as the associated generation-interval parameters, $\bar G$, and $\kappa$) from each study with probability distributions that account for the uncertainty in the estimates;
this allows us to re-estimate the corresponding distributions of the exponential growth rates $r$.
We then use a Bayesian multi-level model to estimate the three  key quantities ($r$, $\bar G$, and $\kappa$).
The resulting pooled estimates ($\mu_r$, $\mu_G$, and $\mu_\kappa$) are used to calculate the pooled estimate of the basic reproduction number, \Rpool.
Using pooled estimates allows us to average appropriately across the uncertainties present in modeling approaches and in their underlying assumptions.''

\revtext{There is still a confusion between credible intervals and confidence
intervals. When the authors use a frequentist approach, they estimate a confidence interval;
while the authors use a bayesian approach they estimate a credible interval. So it is study,
they estimate credible intervals but they estimated intervals they used from the literature can
be both confidence and credible intervals. Therefore, it is case, use the terms
confidence/credible intervals.}

We tried to be as careful as possible in the uses of confidence and credible intervals.
We still describe our base estimates with credible intervals because the base estimates are our Bayesian approximations based on the reported estimates taken from each study; we have tried to make this distinction clear in the text.

\revtext{In this work, all the studies have all the same weight 1/7 when they calculated pooled estimates
cf. Equation 4. They did not give more weight to the studies, which were the more likely}

We now provide a better justification:

``Although this approach nominally treats all studies are equally weighted, the overall pooled estimate will still be weighted by the certainty of the reported estimates (e.g., $r_i$ will be sampled from a narrow distribution and therefore have stronger influence on $\mu_r$ if the reported confidence/credible interval on $r_i$ is narrow).''

\revtext{More specific comments}

\revtext{The authors do not yet give information about the other sources of uncertainty linked
with the variation of parameter sigma ($\sigma_r$, $\sigma_G$, $\sigma_\kappa$).}

We have added the following sentences:

``The estimates of the between-study standard deviations further suggests that there is a large variability in the underlying parameters among the seven studies, particularly in $r$ and $\kappa$:
$\sigma_r = 0.07\,\textrm{days}^{-1}$ (95\% CI: 0.04--0.19 $\textrm{days}^{-1}$),
$\sigma_G = 1.02\,\textrm{days}$ (95\% CI: 0.54--2.50 days),
and
$\sigma_\kappa = 0.51$ (95\% CI: 0.24--1.52).
This variability is likely driven by the differences in modeling approaches and assumptions.''

\revtext{Keep the same order for exemple always $r$, $\bar G$ and $\kappa$.}

Done.

\revtext{
The new figure 2 is nice. Change the caption of the figure 2 “Effects of the uncertainties of $r$, $\bar G$, and $\kappa$ on the estimates of $\Ro$” by “Effects of growth rate ($r$), Mean generation interval ($\bar G$), and Generation-interval dispersion ($\kappa$) on the estimates of $\Ro$”. The caption must be explicit by itself.
}

Done.

\revtext{
The new figure 3 is nice but please represent on figure 3 the value of the R0
and its IC published in the 7 studies (Cf table1) for example in black.
}

Done.

\revtext{
Indicate the number of posterior samples used (2000? or 1000?).
}

Done.

\revtext{
When the authors speak about “credible intervals” from the literature, they must
replace every time by “confidence/credible” because it depends on the methods
of previous studies.
}

Done.

\revtext{
The mutation of the virus or the presence of a new clade of the virus having
different R0 can also affect estimation and inference.
}

We have added the following sentence in the discussion:

``Emergence of a new strain (or a clade) with different transmissibility could also affect disease dynamics, and complicate inference; this study does not address this possibility.''

\end{document}
