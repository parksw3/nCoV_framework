\documentclass[12pt]{article}
\usepackage[top=1in,left=1in, right = 1in, footskip=1in]{geometry}

\usepackage{graphicx}
%\usepackage{adjustbox}

\newcommand{\eref}[1]{(\ref{eq:#1})}
\newcommand{\fref}[1]{Fig.~\ref{fig:#1}}
\newcommand{\Fref}[1]{Fig.~\ref{fig:#1}}
\newcommand{\sref}[1]{Sec.~\ref{#1}}
\newcommand{\frange}[2]{Fig.~\ref{fig:#1}--\ref{fig:#2}}
\newcommand{\tref}[1]{Table~\ref{tab:#1}}
\newcommand{\tlab}[1]{\label{tab:#1}}
\newcommand{\seminar}{SE\mbox{$^m$}I\mbox{$^n$}R}

\usepackage{amsthm}
\usepackage{amsmath}
\usepackage{amssymb}
\usepackage{amsfonts}

%\usepackage{lineno}
%\linenumbers

\usepackage[pdfencoding=auto, psdextra]{hyperref}

\usepackage{natbib}
\bibliographystyle{chicago}
\date{\today}

\usepackage{xspace}
\newcommand*{\ie}{i.e.\@\xspace}

\usepackage{color}

\newcommand{\Rx}[1]{\ensuremath{{\mathcal R}_{#1}}} 
\newcommand{\Ro}{\Rx{0}}
\newcommand{\RR}{\ensuremath{{\mathcal R}}}
\newcommand{\Rhat}{\ensuremath{{\hat\RR}}}
\newcommand{\tsub}[2]{#1_{{\textrm{\tiny #2}}}}

\newcommand{\comment}[3]{\textcolor{#1}{\textbf{[#2: }\textsl{#3}\textbf{]}}}

%% \newcommand{\rev}[1]{\comment{red}{REV}{#1}}
\newcommand{\rev}[1]{}

\newcommand{\swp}[1]{\comment{magenta}{SWP}{#1}}
\newcommand{\new}[1]{\textcolor{blue}{#1}}

\begin{document}

\begin{flushleft}{
	\Large
	\textbf\newline{
		Evaluating uncertainties associated with early estimates of the basic reproduction number during the novel coronavirus (2019-nCoV) outbreak
	}
}
\newline
\\
Sang Woo Park\textsuperscript{1}
Jonathan Dushoff\textsuperscript{2,3}
\\

\bigskip
\textbf{1} Department of Ecology and Evolutionary Biology, Princeton University, Princeton, New Jersey, USA
\\
\textbf{2} Department of Biology, McMaster University, Hamilton, Ontario, Canada
\\
\textbf{3} Michael G. DeGroote Institute for Infectious Disease Research, McMaster University, Hamilton, Ontario, Canada
\\
\bigskip

% *swp2@princeton.edu
\end{flushleft}

\section*{Abstract}

\pagebreak

\section{Introduction}

Since the end of December 2019, a novel coronavirus (2019-nCoV) continues to
spread in China and in other parts of the world.
As of January 27th, 2020, the World Health Organization has
confirmed 2798 cases, including 37 confirmed cases in 11 different
countries, outside China \citep{who27report}.
Although the virus is suspected to have originated
from animal reservoirs, a recent case from Viet Nam
demonstrated its ability to transmit between
humans \citep{who26report},
posing a greater threat for a wider spread.

Many researchers have been rushing to publish their 
analysis of the outbreak \citep{imaincov, riouncov,readncov,zhaoncov,majumderncov,liuncov} 
and, in particular, their
estimates of the basic 
reproductive number $\mathcal R_0$ (i.e., the 
average number of secondary cases generated 
by a primary case in a fully susceptible population).
The basic reproductive number is of particular interest 
because it allows prediction about the final size of an epidemic.
While their efforts are valuable, their analyses rely on several
assumptions that could immediately affect their estimates of $\mathcal R_0$ and
the associated uncertainties.

Here, we present a simple framework for evaluating uncertainties 
associated with parameter estimates
across a wide range of models.
Our results indicate that most published estimates of $\mathcal R_0$
are likely to be overly confident. 
We also lay down several principles that needs to be taken into
consideration.

\section*{Early estimates of $\mathcal R_0$}

Early in an outbreak, $\mathcal R_0$ cannot be estimated directly.
Instead, estimates of $\mathcal R_0$ are often inferred from
the exponential growth rate $r$ can be estimated reliably from incidence data.
Given estimates of the exponential growth rate $r$ and the distribution $g(\tau)$ of 
generation intervals (i.e., the time between when a person become 
infected and that person infects another person), the basic reproduction
number can be estimated via the Euler-Lotka equation:
\begin{equation}
1/\mathcal R_0 = \int \exp(-r\tau) g(\tau) d\tau.
\end{equation}
In other words, estimates of $\mathcal R_0$
necessarily depends on the assumptions about 
growth rate $r$ and the shape of generation-interval distribution $g(\tau)$.

\begin{figure}[t]
\includegraphics[width=\textwidth]{compare_assumption.pdf}
\caption{
Early estimates of $\mathcal R_0$ and associated assumptions about $r$ and $g(\tau)$.
}
\end{figure}

We use the gamma approximation framework to demonstrate
how assumptions about $r$ and $g(\tau)$
affects $\mathcal R_0$.
Assuming that generation intervals follow a gamma distribution 
with the mean $\bar G$ and the squared coefficient of variation $\kappa$, 
we have
\begin{equation}
\mathcal R_0 = \left(1 + \kappa r \bar{G}\right)^{1/\kappa}.
\end{equation}
This equation demonstrates that generation-interval distributions
with a larger mean (higher $\bar{G}$) or less variability (lower $\kappa$)
will give a higher estimate of $\mathcal R_0$.
Likewise, strong assumptions about $\bar{G}$ and $\kappa$ will give
estimates of $\mathcal R_0$ with narrow confidence intervals.

Figure 1 clearly demonstrates the lack of uncertainties in the underlying parameters
of the current $\mathcal R_0$ estimates.
In particular, no studies properly propagate uncertainties associated with the
amount of variability in generation intervals, which can have 
large effects on the estimates of $\mathcal R_0$
For example, when the relative mean generation interval is long ($r\bar{G}\approx 2$ based on the estimates by \cite{zhaoncov}),
varying $\kappa$ from 0 to 1 reduces the estimate of $\mathcal R_0$ by more than a two fold.

\begin{equation}
\begin{aligned}
\mathcal R_i &\sim \textrm{distrib}(\theta_i)\\
\bar G_i &\sim \textrm{distrib}(\theta_i)\\
\kappa &\sim \textrm{distrib}(\theta_i)\\
\end{aligned}
\end{equation}

\bibliography{wuhan}

\end{document}
