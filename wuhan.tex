\documentclass[12pt]{article}
\usepackage[top=1in,left=1in, right = 1in, footskip=1in]{geometry}

\usepackage{graphicx}
%\usepackage{adjustbox}

\newcommand{\eref}[1]{(\ref{eq:#1})}
\newcommand{\fref}[1]{Fig.~\ref{fig:#1}}
\newcommand{\Fref}[1]{Fig.~\ref{fig:#1}}
\newcommand{\sref}[1]{Sec.~\ref{#1}}
\newcommand{\frange}[2]{Fig.~\ref{fig:#1}--\ref{fig:#2}}
\newcommand{\tref}[1]{Table~\ref{tab:#1}}
\newcommand{\tlab}[1]{\label{tab:#1}}
\newcommand{\seminar}{SE\mbox{$^m$}I\mbox{$^n$}R}

\usepackage{amsthm}
\usepackage{amsmath}
\usepackage{amssymb}
\usepackage{amsfonts}

%\usepackage{lineno}
%\linenumbers

\usepackage[pdfencoding=auto, psdextra]{hyperref}

\usepackage{natbib}
\bibliographystyle{chicago}
\date{\today}

\usepackage{xspace}
\newcommand*{\ie}{i.e.\@\xspace}

\usepackage{color}

\usepackage{xspace}
\newcommand{\Rx}[1]{\ensuremath{{\mathcal R}_{#1}}} 
\newcommand{\Ro}{\Rx{0}\xspace}
\newcommand{\RR}{\ensuremath{{\mathcal R}}}
\newcommand{\Rhat}{\ensuremath{{\hat\RR}}}
\newcommand{\tsub}[2]{#1_{{\textrm{\tiny #2}}}}

\newcommand{\comment}[3]{\textcolor{#1}{\textbf{[#2: }\textsl{#3}\textbf{]}}}

%% \newcommand{\rev}[1]{\comment{red}{REV}{#1}}
\newcommand{\rev}[1]{}

\newcommand{\swp}[1]{\comment{magenta}{SWP}{#1}}
\newcommand{\jd}[1]{\comment{magenta}{JD}{#1}}
\newcommand{\bmb}[1]{\comment{magenta}{BMB}{#1}}
\newcommand{\dc}[1]{\comment{magenta}{DC}{#1}}
\newcommand{\new}[1]{\textcolor{blue}{#1}}

\begin{document}

\begin{flushleft}{
	\Large
	\textbf\newline{
		Reconciling early-outbreak preliminary estimates of the basic reproductive number: a new framework and applications to the novel coronavirus (2019-nCoV) outbreak
	}
}
\newline
\\
Sang Woo Park\textsuperscript{?}
David Champredon\textsuperscript{?}
Michael Li\textsuperscript{?}
Bryan T. Grenfell\textsuperscript{?}
Jonathan Dushoff\textsuperscript{?}
\\
\bigskip
\textbf{1} Department of ???
\\
\textbf{2} Department of ???
\\
\textbf{3} Department of ???
\\
\textbf{4} Department of ???
\\
\textbf{5} Department of ???
\\
\bigskip

% *swp2@princeton.edu
\end{flushleft}

\section*{Abstract}

\pagebreak

\section{Introduction}

Since December 2019, the novel coronavirus (2019-nCoV) has been spreading in China and other parts of the world.
As of January 28th, 2020, the World Health Organization has confirmed 4593 cases, including 56 confirmed cases in 14 different countries, outside China \citep{who28report}.
Although the virus is believed to have originated from animal reservoirs \citep{cdcncov}, 
there is clear evidence that it can be directly transmitted between humans,
posing a greater threat for its spread \citep{huang2020clinical,who26report}.

As the disease continues to spread, many researchers have published analyses of the outbreak, focusing in particular on estimates of the basic reproductive number \Ro (i.e., the average number of secondary cases generated by a primary case in a fully susceptible population \citep{anderson1991infectious, diekmann1990definition}).
Estimating the basic reproductive number is of interest during an outbreak because it provides information about the level of intervention required to interrupt transmission \citep{anderson1991infectious}, and about the final size of the outbreak \citep{anderson1991infectious, ma2006generality}.
We congratulate these researchers for their timely contribution and those who made the data publicly available.
However, it can be difficult to assess the validity of the estimates of \Ro (as well as the associated degrees of uncertainty) when the estimation methods and their underlying assumptions vary widely, especially since these assumptions can directly affect the estimates.
\dc{Even if this is not directly addressed here, don’t you also want to talk about observation error/delays, which are clearly present with this outbreak?}

Here, we show that a wide range of approaches to estimating \Ro\ can be understood and compared in terms of estimates for three quantities: the exponential growth rate $r$, the mean generation interval $\bar G$, and the generation-interval dispersion $\kappa$.
The generation interval, which is defined as the time between when an individual becomes infected and when that individual infects another individual \citep{svensson2007note}, plays a key role in shaping the relationship between $r$ and \Ro\ \citep{wearing2005appropriate, roberts2007model, wallinga2007generation, park2019practical};
early in an epidemic, information is scarce and, inevitably, there is large uncertainty around case reports (affecting the estimates of exponential growth rate) and contact tracing (affecting the estimates of generation intervals).
We suggest that disease modelers should make sure their assumptions about these three quantities are clear and reasonable, and that estimates of uncertainty should propagate error from all three sources.

We evaluate six disparate models posted online between January 24--26, 2020 that estimated \Ro\ for the 2019-nCoV outbreak. We use our framework to construct pooled estimates for the underlying parameters. We use these pooled estimates to illustrate the importance of propagating different sources of error, particularly uncertainty in both the growth rate and the generation interval. We also use our framework to unravel which assumptions of these different models led to their different estimates and confidence intervals.

\section{Results}

\begin{table}[t]
\begin{center}
\scriptsize
\begin{tabular}{l|l|l|l|c}
 & Basic reproductive number & Mean generation time (days) & CV$^2$ generation time & Reference \\
\hline
Study 1 & 2.5 (1.5--3.5)$^\ast$ & 8.4 & unspecified$^\dagger$ & \cite{imaincov} \\
\hline
Study 2 & 2.92 (95\% CI: 2.28--3.67) & 8.4 & 0.2 & \cite{liuncov} \\
\hline
Study 3 & 3.8 (95\% CI: 3.6--4.0) & 7.6 & 0.5 & \cite{readncov} \\
\hline
Study 4 & 2.2 (90\% CI: 1.4--3.8) & 7--14 & 0.5 & \cite{riouncov} \\
\hline
Study 5 & 5.47 (95\% CI: 4.16--7.10)$^\ddagger$ & 7.6--8.4 & 0.2 & \cite{zhaoncov} \\
\hline
Study 6 & 2.0--3.1 & 6--10 & 0 & \cite{majumderncov} \\
\hline
\end{tabular}
\end{center}
\caption{
\textbf{Reported estimates of the basic reproductive number and the assumptions about the generation-interval distributions.}
Estimates of \Ro and their assumptions about the shape of the generation interval distributions were collected from 6 studies.
$^\ast$We treat these intervals as a 95\% confidence interval in our analysis.
$^\dagger$We assume $\kappa = 0.5$ in our analysis.
$^\ddagger$The authors presented \Ro estimates under different assumptions; we use their baseline scenario in our analysis.
}
\end{table}

We gathered information on estimates of \Ro and their assumptions about the underlying generation-interval distributions from 6 articles that were published online between January 24th, 2020 and January 26th, 2020 (Table 1).
For each study $i$, we analyze explicitly or implicitly reported distributions of the reproductive number $\mathcal R_{0i}$, the mean generation interval $\bar G_i$, and the generation-interval dispersion parameter $\kappa_i$ (equivalent to the squared coefficient of variation for a Gamma distribution).
We used Gamma distributions to model values reported with confidence intervals and uniform distributions to model values reported with ranges.
For example, Study 2 estimated $\Ro = 2.92$ (95\% CI: 2.28--3.67);
we model this parameter as a Gamma distribution with a mean of 2.92 and a shape parameter of 67, which has a 95\% probability of containing a value between 2.28 and 3.67 (see Table 2 for a complete description).
We then constructed a family of parameter sets for each study, and used these in a Bayesian multilevel model to build a distribution of pooled estimates (see Methods).
\dc{the “uniform” part relates to G, right? If so, it’s not clear
neither here, nor later, only in Table 2}
\swp{This is coincidence... Not sure how to explain it...}

\newcommand{\gammdist}{\mathrm{Gamma}}
\begin{table}[t]
\begin{center}
\footnotesize
\begin{tabular}{l|l|l|l}
 & Basic reproductive number & mean generation time (days) & CV$^2$ generation time \\
\hline
Study 1 & $\gammdist(\alpha=28, \beta=28/2.6)$ & constant & constant \\
\hline
Study 2 & $\gammdist(\alpha=67, \beta=67/2.92)$ & constant & constant \\
\hline
Study 3 & $\gammdist(\alpha=1400, \beta=1400/3.8)$ & constant & constant \\
\hline
Study 4 & $\gammdist(\alpha=12, \beta=12/2.2)$ & $\mathrm{Uniform}(7, 14)$ & constant\\
\hline
Study 5 & $\gammdist(\alpha=54, \beta=54/5.47)$ & $\mathrm{Uniform}(7.6, 8.4)$ & constant\\
\hline
Study 6 & N.A.$^\ast$ & $\mathrm{Uniform}(6, 10)$ & constant\\
\hline
\end{tabular}
\end{center}
\caption{
\textbf{Probability distributions for \Ro, $\bar G$, and $\kappa$.}
We use these probability distributions to obtain a probability distribution for the exponential growth rate $r$.
The Gamma distribution is parameterized by shape $\alpha$ and rate $\beta$ parameters.
Constant distributions assume fixed values according to Table 1.
$^\ast$Study 6 uses the IDEA model \citep{fisman2013idea}, through which the authors effectively fit an exponential curve to the cumulative number of confirmed cases without propagating any statistical uncertainty.
Instead of modeling \Ro with a probability distribution and recalculating $r$, we use $r=0.114\,\mathrm{days}^{-1}$, which explains all uncertainty in \Ro they report, when combined with the range of $\bar G$ they consider.
}
\end{table}

\fref{assumption} compares the reported values of the exponential growth rate $r$, mean generation interval $\bar G$, and the squared coefficient of variation (CV) $\kappa$ of the generation-interval distribution from different studies with our pooled estimates ($\mu_r$, $\mu_G$, and $\mu_\kappa$) that we calculate from our multilevel model.
We find that there is a large uncertainty associated with the underlying parameters;
many models rely on stronger assumptions that ignore these uncertainties.
Surprisingly, no studies take into account how the variation in generation intervals affects their estimates of \Ro:
all studies assumed fixed values for $\kappa$, ranging from 0 to 0.5.
Assuming fixed parameter values can lead to overly strong conclusions \citep{elderd2006uncertainty}.
It is also interesting that none of the six studies explicitly or implicitly assumed an exponentially distributed generation interval (i.e., $\kappa=1$).

\begin{figure}[t]
\includegraphics[width=\textwidth]{compare_assumption.pdf}
\caption{
\textbf{Comparisons of the reported parameter values with our pooled estimates.}
We inferred point estimates, uniform distributions (shown in orange) or confidence intervals (purple) for each parameter from each study, and combined them into pooled estimates (see text).
Black points and lines: reported parameter values and ranges of uncertainties considered.
Black triangle: we assumed $\kappa=0.5$ for Study 1 as they do not report the parameter values that they used.
Red points and lines: median and 95\% confidence intervals of our pooled estimates.
}
\label{fig:assumption}
\end{figure}

\fref{eff} shows how propagating uncertainty ($\mu_r$, $\mu_G$, and $\mu_\kappa$) in different combinations would affect estimates and CIs for our pooled estimates, which represent a reasonable proxy for the state of knowledge as of 26 January. 
Propagating error from the growth rate (which all but one of the models did) is absolutely crucial:
uncertainty on \Ro, measured with our 95\% CI, is more than two fold in this case (95\% CI: 2.22--5.16).
Propagating error from the generation interval also has important effects, however (95\% CI: 2.59--4.12).
Once these two are included, the impact of leaving out uncertainty in the dispersion would be small, in this particular example -- 
propagating errors from $r$ and $\bar G$ gives 95\% CI of 2.16--5.58 whereas propagating all errors gives 95\% CI of 2.06--5.89 (compare \textbf{growth rate + GI mean} with \textbf{all} in \fref{eff}).

\begin{figure}[!ht]
\includegraphics[width=\textwidth]{figure2.pdf}
\caption{
\textbf{Effects of $r$, $\bar G$, and $\kappa$ on the estimates of \Ro.}
We compare estimates of \Ro under five scenarios that propagate different combinations of uncertainties.
\textbf{base}: \Ro estimates based on the median estimates of $\mu_r$, $\mu_G$, and $\mu_\kappa$.
\textbf{growth rate}: \Ro estimates based on the the posterior distribution of $\mu_r$ while using median estimates of $\mu_G$ and $\mu_\kappa$.
\textbf{GI mean}: \Ro estimates based on the the posterior distribution of $\mu_G$ while using median estimates of $\mu_r$ and $\mu_\kappa$.
\textbf{growth rate + GI mean}: \Ro estimates based on the the joint posterior distributions of $\mu_r$ and $\mu_G$ while using a median estimate of $\mu_\kappa$.
\textbf{all}: \Ro estimates based on the joint posterior distributions of  $\mu_r$, $\mu_G$, and $\mu_\kappa$.
Vertical lines represent the 95\% confidence intervals.
}
\label{fig:eff}
\end{figure}

We also evaluate the estimates of \Ro across different studies by 
replacing their values of $r$, $\bar G$, and $\kappa$ with our pooled estimates ($\mu_r$, $\mu_G$, and $\mu_\kappa$) one at a time and recalculating the basic reproductive number \Ro (\fref{R0}).
We find that incorporating uncertainties one at a time increases the width of the confidence intervals all but three cases.
We estimate slightly narrower confidence intervals for Study 2 and Study 6 when we use our pooled estimate of the squared CV in generation time $\mu_\kappa$ to recalculate \Ro because they assume a narrow generation-interval distribution (compare \textbf{base} with \textbf{GI variation} in \fref{R0});
when higher values of $\kappa$ are used, their estimates of \Ro become less sensitive to the values of $r$ and $\bar G$, giving narrower confidence intervals.
We estimate narrower confidence intervals for Study 4 when we use our pooled estimate of the mean generation time $\mu_G$ to recalculate \Ro (compare \textbf{base} with \textbf{GI mean} in \fref{R0}) because the range of uncertainty in the mean generation time $\bar G$ they consider is much wider than ours (\fref{assumption}).

Consistent with our previous observations (\fref{eff}),
we find that accounting for uncertainties in the estimate of $r$ has the largest effect on the estimates of \Ro (\fref{R0}).
For example, recalculating \Ro for Study 6 by using our pooled estimate of $r$ gives $\mathcal R_0 = 4.0$ (95\% CI: 2.3--10.1), which is much wider than the range they reported (2.0--3.1).
There are two explanations for this result.
First, even though the exponential growth rate $r$ and the mean generation time $\bar G$ have identical effects on \Ro under the gamma approximation framework \eref{gamma},
$r$ has a greater overall effect on \Ro because it is associated with more uncertainty;
the range of the 95\% confidence interval on $\mu_r$ spans over more than a two-fold difference, whereas the range of the 95\% confidence intervals on $\mu_G$ does not.
Second, assuming a fixed generation time ($\kappa=0$) makes the estimate of \Ro too sensitive to $r$ and $\bar G$ as explained previously.

\begin{figure}[!th]
\includegraphics[width=\textwidth]{compare_R0.pdf}
\caption{
\textbf{Sensitivity of the reported \Ro estimates with respect to our pooled estimates of the underlying parameters.}
We replace the reported parameter values (growth rate $r$, GI mean $\bar G$, and GI variation $\kappa$) with our corresponding pooled estimates ($\mu_r$, $\mu_G$, and $\mu_\kappa$) one at a time and recalculate \Ro (\textbf{growth rate}, \textbf{GI mean}, and \textbf{GI variation}).
The pooled estimate of \Ro is calculated from the joint posterior distribution of $\mu_r$, $\mu_G$, and $\mu_\kappa$ (\textbf{all});
this corresponds to replacing all reported parameter values with our pooled estimates, which gives identical results across all studies.
Horizontal dashed lines represent the 95\% confidence intervals of our pooled estimate of \Ro.
The reported \Ro estimates (\textbf{base}) have been adjusted to show the approximate 95\% confidence interval using the probability distributions that we defined if they had relied on different measures for parameter uncertainties.
}
\label{fig:R0}
\end{figure}

Finally, we incorporate all uncertainties by using posterior samples for $\mu_r$, $\mu_G$, and $\mu_\kappa$ to recalculate \Ro and compare it with the reported \Ro estimates.
We estimate the average \Ro to be 3.1 (95\% CI: 2.1--5.7).
While our pooled estimate of \Ro is similar to other reported values, our calculation gives wider confidence intervals than all of them.
This result does not imply that assumptions based on the pooled estimate are too weak;
we believe that this confidence interval accurately reflects the level of uncertainties present in the information that was available when these models were fitted.

\section{Discussion}

Estimating the basic reproductive number \Ro is crucial for predicting the course of an outbreak and planning intervention strategies.
Here, we used a simple framework \citep{park2019practical} to compare estimates of \Ro for the novel coronavirus outbreak.
Our results demonstrate the importance of accounting for uncertainties associated with the underlying generation-interval distributions, including with the amount of dispersion in the generation intervals:
although our pooled estimates are relatively insensitive to the estimated uncertainty, our analysis of individual studies shows that assuming too narrow a generation-interval distribution can make the estimate of \Ro too sensitive to the estimates of the exponential growth rate $r$.

Narrow confidence intervals associated with early estimates of \Ro partly depend on the statistical approaches used to estimate the exponential growth rate $r$.
Of the six studies that we reviewed, two of them directly fit their models to cumulative number of confirmed cases.
Although the formula for the cumulative curve (e.g., logistic model) is much simpler and familiar to many people, fitting a model to cumulative incidence, instead of raw incidence, can both bias parameter and give narrow confidence intervals if non-independent error structures are not taken into account \citep{ma2014estimating, king2015avoidable}.
Naive fits to cumulative incidence data should be avoided.

More broadly, disease modelers should always consider uncertainties in the data generating processes.
Process vs obs noise.

Our results indicate that the uncertainty in the estimates of the exponential growth rate $r$ matters the most.
However, as the epidemic progresses and more data becomes available, inference on will become more precise, eventually decreasing.
Uncertainty in the exponential growth rate will decrease, 
and the effect of uncertainty in the mean generation time will become more important.

We have provided a basis for evaluating and comparing exponential-growth based estimates of \Ro\ in terms of three simple components. We are hopeful that this will provide a guide to understanding and reconciling different estimates early in an epidemic.

\section{Methods}

\subsection{Gamma approximation framework for linking $r$ and $\mathcal R_0$}

Early in an outbreak, \Ro cannot be estimated directly;
instead, \Ro is often inferred from the exponential growth rate $r$, which can be estimated reliably from incidence data \citep{mills2004transmissibility, nishiura2009transmission, ma2014estimating}.
Given an estimate of the exponential growth rate $r$ and an \emph{instrinsic} generation-interval distribution $g(\tau)$ \citep{champredon2015intrinsic}, the basic reproductive
number can be estimated via the Euler-Lotka equation \citep{wallinga2007generation}:
\begin{equation}
1/\mathcal R_0 = \int \exp(-r\tau) g(\tau) d\tau.
\label{eq:euler}
\end{equation}
In other words, estimates of \Ro must
depend on the assumptions about the
exponential growth rate $r$ and the shape of the generation-interval distribution $g(\tau)$.

Here, we use the gamma approximation framework \citep{mcbryde2009early, nishiura2009transmission, roberts2011early, park2019practical} to (1) characterize the amount of uncertainty present in the exponential growth rates and the shape of the generation-interval distribution and (2) assess the degree to which these uncertainties affect the estimate of \Ro.
Assuming that generation intervals follow a gamma distribution
with the mean $\bar G$ and the squared coefficient of variation (CV$^2$) $\kappa$, 
we have
\begin{equation}
\mathcal R_0 = \left(1 + \kappa r \bar{G}\right)^{1/\kappa}.
\label{eq:gamma}
\end{equation}
This equation demonstrates that a generation-interval distribution
that has a larger mean (higher $\bar{G}$) or is less variable (lower $\kappa$)
will give a higher estimate of \Ro for the same value of $r$.

\subsection{Description of the studies}

We reviewed six modeling studies of the novel coronavirus outbreak that were published online between January 24th, 2020 and January 26th, 2020 (Table 1).
Five studies \citep{liuncov, majumderncov, readncov, riouncov, zhaoncov} were upload on preprint servers (bioRxiv, medRxiv, and SSRN), and one report was published from Imperial College London \citep{imaincov}.
There is a wide variation in their statistical methods and the amount of data they used to infer \Ro.
\cite{imaincov} and \cite{riouncov} simulated branching process models and compared the predicted number of cases from their models with the estimated number of total cases by January 18th.
\cite{readncov} fitted a deterministic, metapopulation Susceptible-Exposed-Infected-Recovered (SEIR) model to incidence data between January 1st and January 21st from major cities in China and other countries.
\cite{zhaoncov} and \cite{liuncov} fitted exponential growth models to incidence data up to January 22nd and January 23rd, respectively, and inferred \Ro\ via the Euler-Lotka equation \eref{euler}.
\cite{majumderncov} fitted the Incidence Decay and Exponential Adjustment (IDEA) model \citep{fisman2013idea} to incidence data up to January 26th, which is equivalent to fitting an exponential growth model and assuming a fixed generation-interval distribution.

\subsection{Statistical framework}

For each study, we construct a family of parameter sets by drawing 100,000 random samples from the probability distributions (Table 2) that represent the estimates of \Ro and the assumed values of $\bar G$ and $\kappa$ and calculating the exponential growth rate $r$ via the inverse of \eref{gamma}:
\begin{equation}
r_i = \frac{{\mathcal R_{0i}}^{\kappa_i} - 1}{\kappa_i \bar{G}_i}.
\end{equation}
This allows us to approximate the probability distributions of the estimated exponential growth rates by each study;
uncertainties in the probability distributions that we calculate for the estimated exponential growth rates will reflect the methods and assumptions that the studies rely on.

\dc{I think I understand why you do that. But I’m worried the uncertainty for r comes
artificially from whatever methods the Studies used, not the actual observation process we
would have by observing (imperfectly) directly the cases. I guess the studies have all more or
less the same incidence data... why not estimating uncertainty for r directly from incidence}
\swp{that is exactly the point... we want to be able to reflect their methods... added a sentence about it.}


We construct pooled estimates for each parameter ($r$, $\bar G$, and $\kappa$) using a Bayesian multilevel modeling approach, which assumes that the parameters across different studies come from the same gamma distribution:
\begin{equation}
\begin{aligned}
r_i &\sim \gammdist(\mathrm{mean}=\mu_r, \mathrm{shape}=\alpha_r),\\
\bar{G}_i &\sim \gammdist(\mathrm{mean}=\mu_G, \mathrm{shape}=\alpha_G),\\
\kappa_i &\sim \gammdist(\mathrm{mean}=\mu_\kappa, \mathrm{shape}=\alpha_\kappa).\\
\end{aligned}
\end{equation}
We account for uncertainties associated with $r_i$, $\bar G_i$ and $\kappa_i$ (and their correlations), by drawing a random set from the family of parameter sets for each study at each Metropolis-Hastings step;
this approach is analogous to Bayesian methods for analyzing phylogenetic data, which often rely on drawing random samples of phylogenetic trees from a discrete set to account for phylogenetic uncertainty \citep{pagel2004bayesian,bedford2014integrating}.
Since the gamma distribution does not allow zeros, we use $\kappa =0.02$ instead for Study 6.

Weakly informative priors are assumed on the hyperparameters:
\begin{equation}
\begin{aligned}
\mu_r &\sim \gammdist(\mathrm{mean}=1\,\mathrm{week}^{-1},\,\mathrm{shape}=0.1)\\
\mu_G &\sim \gammdist(\mathrm{mean}=1\,\mathrm{week},\,\mathrm{shape}=0.1)\\
\mu_\kappa &\sim \gammdist(\mathrm{mean}=0.5,\,\mathrm{shape}=0.1)\\
(\alpha_r, \alpha_G, \alpha_\kappa) &\sim \gammdist(\mathrm{mean}=1,\,\mathrm{shape}=0.1).
\end{aligned}
\end{equation}
We run 4 parallel Markov Chain Monte Carlo (MCMC) chains that consist of 200,000 burnin steps and 200,000 sampling steps.
Posterior samples are thinned every 400 steps.
Convergence is assessed by ensuring that the Gelman-Rubin statistic is below 1.01 for all hyperparameters \citep{gelman1992inference}.
95\% confidence intervals are calculated by taking 2.5\% and 97.5\% quantiles from the posterior distribution.

\bmb{general comments (Daniel may have taken notes on these already):
  \begin{itemize}
  \item (\textbf{later on}) try flatter-at-origin priors (e.g. half-Cauchy)? Stan implementation?
  \item comment on broader uncertainty context (King/Rohani on observation + process error, major reporting irregularities, non-stationarity \ldots)
  \end{itemize}
}

\section*{Acknowledgements}

We thank Ben Bolker for providing useful comments on the manuscript.

\pagebreak

\bibliography{wuhan}

\end{document}
